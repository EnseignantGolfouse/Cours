\documentclass[a4paper,11pt,landscape,twocolumn]{article}

\usepackage{../préambule}
\usepackage{clipboard}
\usetikzlibrary{calc}

\newcommand{\myAdd}[2]{\directlua{tex.print(#1 + #2)}}

\begin{luacode}
	function print_items(points)
		for _, p in ipairs(points) do
			tex.sprint("\\item (", p.x, "; ", p.y, ")")
		end
	end

	function print_path(points)
		tex.sprint("\\draw[red,thick] ")
		for i, p in ipairs(points) do
		 	if i ~= 1 then
		 		tex.sprint(" -- ")
		 	end
		 	tex.sprint("(", p.x / 2 + 3.5, ",", p.y / 2 + 2.5, ")")
		end
		tex.sprint(";")
	end
\end{luacode}

\begin{document}

\Copy{beta}{
	\directlua{
		points = {
				{ x = -7, y = 6 },
				{ x = -2, y = 1 },
				{ x = -2, y = -2 },
				{ x = -6, y = -2 },
				{ x = -6, y = -4 },
				{ x = 0, y = -4 },
				{ x = 0, y = 0 },
				{ x = 3, y = 0 },
				{ x = 6, y = 3 },
				{ x = 6, y = 6 },
				{ x = 7, y = 6 },
			}
	}

	{\large \textbf{Coordonnées γ :}}
	\begin{multicols}{2}
		\begin{enumerate}[label={\Alph* =}]
			\directlua{print_items(points)}
		\end{enumerate}
	\end{multicols}

	\vspace{2em}
	\hrule
	\vspace{2em}

	\textbf{Labyrinthe γ :} \vspace{1em}

	\begin{tikzpicture}[scale=1.3]
		\draw[gray,ultra thin,->] (-1,2.5) -- (8,2.5);
		\draw[gray,ultra thin,->] (3.5,-1) -- (3.5,6);
		\node[below left] at (3.5,2.5) {0};

		\foreach \x in {-6,-4,-2,2,4,6} {
				\draw[very thin] (\directlua{tex.sprint(\x / 2 + 3.5)},2.5) -- (\directlua{tex.sprint(\x / 2 + 3.5)},2.35) node[below] {\scriptsize \x};
			}

		\foreach \y in {-4,-2,2,4,6} {
				\draw[very thin] (3.5,\directlua{tex.sprint(\y / 2 + 2.5)}) -- (3.35,\directlua{tex.sprint(\y / 2 + 2.5)}) node[left] {\scriptsize \y};
			}

		\draw[ultra thick] (1,2) -- (2,2) -- (2,3) -- (0,5) -- (0,0) -- (7,0) -- (7,5);
		\draw[ultra thick] (5,4) -- (2,4) -- (0,6) -- (7,6);
		\draw[ultra thick] (2,6) -- (3,5) -- (6,5) -- (6,4) -- (5,3) -- (3,3) -- (3,1) -- (1,1);
		\draw[ultra thick] (4,0) -- (4,2);
		\draw[ultra thick] (5,2) -- (5,1) -- (6,1) -- (6,3);

		% \directlua{print_path(points)}
	\end{tikzpicture}
}

\newpage

\Paste{beta}

\end{document}