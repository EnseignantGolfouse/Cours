\documentclass[a4paper,11pt]{article}

\usepackage{../../préambule}

\makeatletter
\renewcommand{\maketitle}{%
{\scriptsize colle dans ton cahier d'exercices}
	\begin{center}
		\LARGE
		\myuline{\@title}
		\vspace{0.5em}
	\end{center}
}
\makeatother



\title{Activité : Grille}
\date{}
\author{}

\begin{document}

\maketitle

\warningbox{
	Spécifier l'origine ainsi que les axes.
}

\begin{exercice}
	Mireille a 30€ de côté, un paquet coûte ??? (position abscisse=1, ordonnée = 3).

	Combien de paquets de bonbons doit acheter Mireille pour qu'il lui reste 2 euros ?

	\notebox{
		Réponse:

		7€
	}
\end{exercice}

\begin{exercice}
	Aller au symétrique de (-2,-1) par rapport à (0,1).


	Fait le produit de la somme de l'abscisse et de l'ordonnée trouvée avec le nombre trouvé.
	\notebox{
		Réponse:

		(2,3), nombre trouvé = 6 (résultat = (2 + 3) × 6 = 30)
	}
\end{exercice}

\begin{exercice}
	
\end{exercice}

\end{document}