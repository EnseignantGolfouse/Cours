\documentclass[a4paper,11pt,landscape,twocolumn]{article}

\usepackage{../préambule}
\usepackage{clipboard}

\makeatletter
\renewcommand{\maketitle}{%
{\scriptsize colle dans ton cahier d'exercices}
	\begin{center}
		\LARGE
		\myuline{\@title}
		\vspace{0.5em}
	\end{center}
}
\makeatother

\title{Activité : Labyrinthe}
\date{}
\author{}

\begin{document}

\Copy{Activité}{
	\maketitle

	Voici un jeu qui se joue à 2 : une personne (\textbf{l'indic'}) donne des instructions, pendant que l'autre (\textbf{le prisonnier}) doit sortir d'un labyrinthe.

	Chacun des 2 joueurs à une fiche qui indique ce qu'ils doivent faire.

	\warningbox{
		Il ne faut pas regarder la fiche de l'autre !
	}

	\section*{Pour l'indic'}

	Donne les points, un par un, au prisonnier. Pense bien à dire \textbf{abscisse} et \textbf{ordonnée}, afin de bien de faire comprendre de ton camarade !

	\section*{Pour le prisonnier}

	Tu dispose d'un repère, sur lequel est dessiné un labyrinthe.

	Lorsque l'indic' te donne les coordonnées d'un point, place ce point sur le repère.

	Puis, relie tous les points dans l'ordre dans lequel ils t'ont été donnés pour savoir quel chemin emprunter !

	\vspace{3em}

	\awesomebox[black]{2pt}{\faRocket}{black}{
		Lorque vous finissez un labyrinthe, vous pouvez demander le suivant au professeur.
	}
}

\newpage

\Paste{Activité}

\end{document}