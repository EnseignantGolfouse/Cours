\documentclass[a4paper,11pt]{article}

\usepackage{../préambule}
\usepackage{relsize}
\usetikzlibrary{arrows,positioning,arrows.meta}

\title{Chapitre 5 : Nombres relatifs}
\date{}
\author{}

\begin{document}

\maketitle

\section{Définition des nombres relatifs}

\begin{cours}
	\begin{itemize}
		\item Un nombre \textbf{positif} est un nombre supérieur à 0. On le note avec le signe +, ou sans signe.
		\item Un nombre \textbf{négatif} est un nombre inférieur à 0. On le note avec le signe −.
		\item Les nombres positifs et négatifs forment les nombres \textbf{relatifs}.
	\end{itemize}
\end{cours}

\begin{exemple}
	\begin{itemize}
		\item $3{,}2$ est un nombre positif. On peut aussi le noter $+3{,}2$.
		\item $-5{,}3$ est un nombre négatif.
		\item $0$ est le seul nombre à la fois positif et négatif.
		\item Tous ces nombres ($3{,}2$, $-5{,}3$, $0$, et d'autres) sont des nombres relatifs.
	\end{itemize}
\end{exemple}

\section{Repérage sur une droite}

\begin{definition}[Droite graduée]
	Une \textbf{droite graduée} est une droite sur laquelle on a placé :
	\begin{itemize}
		\item Un point qu'on appelle une \textbf{\color{red} origine}, qui porte le nombre $0$ ;
		\item Un \textbf{\color{Green} sens}, représenté par une flèche ;
		\item Une \textbf{\color{blue} unité de longueur}, qu'on utilise pour marquer de nouveaux points à intervalles réguliers depuis l'origine.
	\end{itemize}

	\begin{tikzpicture}[scale=2]
		\draw[-{Latex[length=3mm, width=2mm]}] (-2.3,0) -- (3.2,0);
		\foreach \p in {-2,-1,0,1,2,3} {
				\draw (\p,0) -- (\p,-0.1) node[below] {$\p$};
			}
		\draw[red,->] (0,0.5) node[above] {Origine} -- (0,0.2);
		\draw[blue,<->] (0,-0.5) -- node[below] {Unité de longueur} (1,-0.5);
		\draw[Green,->] (2.7,0.5) node[above] {Sens} -- ++(0.3,-0.3);
	\end{tikzpicture}
\end{definition}

\newpage

\begin{cours}
	Chaque point d'une droite graduée correspond à un nombre relatif. On l'appelle \textbf{l'abscisse} de ce point.
\end{cours}

\section{Repérage dans un plan}

\section{Comparaison de nombres relatifs}

\section*{Bonus : hiérarchie des nombres}

On remarque que, avec les nombres relatifs, on a ajouté une nouvelle catégories de nombres !

Il existe ainsi plusieurs catégories de nombres, chacune ajoutant un nouveau \textit{type} de nombre :

\begin{itemize}
	\item Les nombres entiers, dits \textbf{naturels}. Ceux-ci contiennent $0, 1, 2, ⋯$.
	\item Les nombres entiers \textbf{relatifs}, qui contiennent $0, 1, 2, ⋯$ mais aussi $-1, -2, -3, ⋯$.

	      On note que strictement parlant, on applique le termes relatifs \textit{aux entiers seulement}, contrairement au cours.
	\item Les nombres \textbf{décimaux} : ce sont les nombres à virgules, mais qui ont seulement un nombre fini de chiffres après la virgule. Par exemple, $2{,}1$ , $5$ ou encore $-6{,}8$.
	\item Les nombres \textbf{rationnels} : ce sont les fractions.
	\item Les nombres \textbf{réels} : ce sont tous les nombres qui peuvent se marquer sur une droite. Par exemple, pi (π) n'est pas un nombre rationnel (il ne peut pas s'écrire sous forme de fraction), mais c'est un nombres réel.
\end{itemize}

On peut schématiser cela par le diagramme suivant :

\begin{center}
	\begin{tikzpicture}[every node/.style={scale=0.85}]
		\draw (0,0) ellipse (2 and 1);
		\draw (0,0.5) ellipse (3.7 and 2);
		\draw (0,1) ellipse (5.4 and 3);
		\draw (0,1.5) ellipse (7.1 and 4);
		\draw (0,2) ellipse (8.8 and 5);

		\node at (0,0.7) {naturels};
		\node at (-0.9,0.1) {0};
		\node at (0,0.1) {1};
		\node at (0.9,0.1) {2};
		\node at (-0.6,-0.5) {50};
		\node at (0.6,-0.5) {3014};

		\node at (0,2.2) {relatifs};
		\node at (-1,1.7) {-1};
		\node at (1.2,1.7) {-76};
		\node at (-2.5,0.5) {-2689};

		\node at (0,3.7) {décimaux};
		\node at (-3,2.7) {0{,}8};
		\node at (2,2.8) {-89{,}127};

		\node at (0,5.2) {rationnels};
		\node at (-6,2) {$-\frac{6}{5}$};
		\node at (-3,4) {$\frac{1}{3}$};
		\node at (5,3.5) {$\frac{60}{859}$};

		\node at (0,6.7) {réels};
		\node at (-5,5) {$\sqrt{3}$};
		\node at (-2,6) {$\sqrt{2}$};
		\node at (3,6) {$π$};
		\node at (8,3) {$e$};
	\end{tikzpicture}
\end{center}

\end{document}