\documentclass[a4paper,11pt]{article}

\usepackage{../préambule}
\usetikzlibrary{matrix,arrows,positioning}

\title{Chapitre 5 : Nombres relatifs}
\date{}
\author{}

\begin{document}

\maketitle

\section{Définition des nombres relatifs}

\begin{cours}
	\begin{itemize}
		\item Un nombre \textbf{positif} est un nombre supérieur à 0. On le note avec le signe +, ou sans signe.
		\item Un nombre \textbf{négatif} est un nombre inférieur à 0. On le note avec le signe −.
		\item Les nombres positifs et négatifs forment les nombres \textbf{relatifs}.
	\end{itemize}
\end{cours}

\begin{exemple}
	\begin{itemize}
		\item $3,2$ est un nombre positif. On peut aussi le noter $+3,2$.
		\item $-5,3$ est un nombre négatif.
		\item $0$ est le seul nombre à la fois positif et négatif.
		\item Tous ces nombres ($3,2$, $-5,3$, $0$, et d'autres) sont des nombres relatifs.
	\end{itemize}
\end{exemple}

\end{document}