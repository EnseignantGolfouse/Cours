\documentclass[a4paper,11pt]{article}

\usepackage{../préambule}
\usepackage{clipboard}
\usetikzlibrary{arrows.meta}

\begin{document}

\Copy{droite}{
	\begin{definition}[Droite graduée]
		Une \textbf{droite graduée} est une droite sur laquelle on a placé :
		\begin{itemize}
			\item Un point qu'on appelle une \textbf{\color{red} origine}, qui porte le nombre $0$ ;
			\item Un \textbf{\color{Green} sens}, représenté par une flèche ;
			\item Une \textbf{\color{blue} unité de longueur}, qu'on utilise pour marquer de nouveaux points à intervalles réguliers depuis l'origine.
		\end{itemize}

		\begin{tikzpicture}[scale=2]
			\draw[-{Latex[length=3mm, width=2mm]}] (-2.3,0) -- (3.2,0);
			\foreach \p in {-2,-1,0,1,2,3} {
					\draw (\p,0) -- (\p,-0.1) node[below] {$\p$};
				}
			\draw[red,->] (0,0.5) node[above] {Origine} -- (0,0.2);
			\draw[blue,<->] (0,-0.5) -- node[below] {Unité de longueur} (1,-0.5);
			\draw[Green,->] (2.7,0.5) node[above] {Sens} -- ++(0.3,-0.3);
		\end{tikzpicture}
	\end{definition}
}

\Paste{droite}

\Paste{droite}

\end{document}