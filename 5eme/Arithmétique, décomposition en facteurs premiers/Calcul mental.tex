\documentclass[a4paper]{beamer}

\usepackage[utf8]{inputenc}     % support des caractères spéciaux (é, þ, ...)
\usepackage{fontspec}           % Pour changer la police
\defaultfontfeatures{Scale=1.4} % text scale
\setmainfont[Ligatures=TeX]{FreeSerif} % Police avec support pour l'UTF-8
\usepackage{unicode-math}       % Symboles unicodes en mode math
\usepackage[francais]{babel}    % françisation partielle de l'output
\usepackage{natbib}             % bibliographie
\usepackage{amsthm}             % maths
\usepackage{amsmath}            % maths
\usepackage{mathtools}          % maths
\usepackage{graphicx}           % figures
\usepackage{titlesec}           % permet de changer l'affichage des sections et autres
\usepackage{color}              % couleurs rgb
\usepackage[tikz]{mdframed}     % boites fancy autour des théorèmes et autres
\usepackage{contour}            % Pour souligner
\usepackage{ulem}               % Pour souligner
\usetikzlibrary{shadows}        % effet d'ombre sur les boites ('shadow = true')
\usepackage{nopageno}           % pas de numéros de page
\usepackage[makeroom]{cancel}   % texte barré
\usepackage{hyperref}           % références
\usepackage{multicol}           % avoir le texte en plusieurs colonnes
\usepackage{caption}            % Custom image captions
\usepackage{floatrow}           % ??? plusieurs images côte à côte
\usepackage{pstricks-add}       % figures ( en l'occurence des droites graduées)
\usepackage[thinlines]{easytable}

% unité de longueur pour pstricks
\psset{unit=0.4cm}

% renommage de la table des matières
\addto\captionsfrench{\renewcommand*\contentsname{Nom de la table des matières}}
\setcounter{tocdepth}{1}

\newcommand{\somme}[2]{\overset{#2}{\underset{#1}{\sum}}}
\newcommand{\produit}[2]{\overset{#2}{\underset{#1}{\prod}}}
\newcommand{\sommedirecte}[2]{\overset{#2}{\underset{#1}{\oplus}}}
\renewcommand*{\proofname}{Preuve}

\renewcommand{\ULdepth}{1.8pt}
\contourlength{0.8pt}

% Commande de soulignage
\newcommand{\myuline}[1]{%
  \uline{\phantom{#1}}%
  \llap{\contour{white}{#1}}%
}

% Pour entourer un nombre
\newcommand*\circled[1]{\tikz[baseline=(char.base)]{
            \node[shape=circle,draw,inner sep=2pt] (char) {#1};}}


%%%%%%%%%%%%%%%%%%

\title{Activité : alignements célestes}
\date{}

%%%%%%%%%%%%%%%%%%

\begin{document}

\begin{frame}
	$$
		\begin{array}{r|r}
			16 & 3 \\
			\cline{2-2}
			   &
		\end{array}
	$$
\end{frame}

\begin{frame}
	$$
		\begin{array}{r|r}
			16           & 3               \\
			\cline{2-2}
			\circled{1}  & \circled{5}     \\
			\text{reste} & \text{quotient}
		\end{array}
	$$
\end{frame}

\begin{frame}
	$$
		\begin{array}{r|r}
			17 & 5 \\
			\cline{2-2}
			   &
		\end{array}
	$$
\end{frame}

\begin{frame}
	$$
		\begin{array}{r|r}
			17           & 5               \\
			\cline{2-2}
			\circled{2}  & \circled{3}     \\
			\text{reste} & \text{quotient}
		\end{array}
	$$
\end{frame}

\begin{frame}
	$$
		\begin{array}{r|r}
			30 & 7 \\
			\cline{2-2}
			   &
		\end{array}
	$$
\end{frame}

\begin{frame}
	$$
		\begin{array}{r|r}
			30           & 7               \\
			\cline{2-2}
			\circled{2}  & \circled{4}     \\
			\text{reste} & \text{quotient}
		\end{array}
	$$
\end{frame}

\begin{frame}
	$$ 6 × ?? = 24 $$
\end{frame}

\begin{frame}
	$$ 6 × \circled{4} = 24 $$
\end{frame}

\begin{frame}
	$$ 5 × ?? = 35 $$
\end{frame}

\begin{frame}
	$$ 5 ×\circled{7} = 35 $$
\end{frame}

\begin{frame}
	$$ 8 × ?? = 72 $$
\end{frame}

\begin{frame}
	$$ 8 × \circled{9} = 72 $$
\end{frame}

\end{document}