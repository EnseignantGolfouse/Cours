\documentclass[a4paper]{article}

\usepackage{../../préambule}

%%%%%%%%%%%%%%%%%%

\title{Chapitre 1 : multiples, diviseurs, nombres premiers}
\date{}
\author{}

%%%%%%%%%%%%%%%%%%


\begin{document}

\maketitle

\begin{cours}
	Si on a trois nombres $a$, $b$ et $c$ tels que
	$$ \textcolor{blue}{a} × \textcolor{blue}{b} = \textcolor{red}{c} $$
	On dit que
	\begin{itemize}
		\setlength\itemsep{0.3em}
		\item $a$ et $b$ sont des \textcolor{blue}{diviseurs} de $c$.
		\item $c$ est un \textcolor{red}{multiple} de $a$ et de $b$.
		\item On dit que $c$ est \myuline{divisible} par $a$ et $b$.
	\end{itemize}
\end{cours}

\begin{exemple}
	\begin{itemize}
		\setlength\itemsep{0.3em}
		\item 2 est un \textcolor{blue}{diviseur} de 6.
		\item 7 est un \textcolor{blue}{diviseur} de 21.
		\item 6 est un \textcolor{red}{multiple} de 2.
		\item 6 est un \textcolor{red}{multiple} de 3.
		\item 48 est un \textcolor{red}{multiple} de 4.
	\end{itemize}
\end{exemple}

\begin{cours}[Critères de divisibilité] \label{cours:criteres-de-divisibilite}
	Parfois, on peut rapidement savoir si un nombre est un diviseur d'un autre nombre. \\
	\begin{itemize}
		\setlength\itemsep{0.3em}
		\item Si le dernier chiffre d'un nombre est pair, ce nombre est un multiple de 2.
		\item Si la somme des chiffres d'un nombre est un multiple de 3, ce nombre est un multiple de 3.
		\item Si le dernier chiffre d'un nombre est 0 ou 5, ce nombre est un multiple de 5.
		\item Si la somme des chiffres d'un nombre est un multiple de 9, ce nombre est un multiple de 9.
	\end{itemize}
\end{cours}

\begin{exemple}
	\begin{itemize}
		\setlength\itemsep{0.3em}
		\item 1244 est un multiple de 2, car 4 est un multiple de 2.
		\item 546 est un multiple de 3, car $5 + 4 + 6 = 15$, qui est un multiple de 3.
		\item 200, 15, 35... Sont des multiples de 5.
		\item 279 est un multiple de 9, car $2 + 7 + 9 = 18$ est un multiple de 9.
	\end{itemize}
\end{exemple}

\begin{cours}
	Une \myuline{division euclidienne} se fait entre deux nombres entiers $a$ et $b$. Il en résulte un \myuline{quotient} et un \myuline{reste}.
	$$
		\begin{array}{r|r}
			a            & b               \\
			\cline{2-2}
			\vdots       & \text{quotient} \\
			\text{reste} &
		\end{array}
	$$

	$$ a = b × \text{quotient} + \text{reste} $$

	On obtient le quotient en soustrayant $b$ aux chiffres de $a$.
\end{cours}

\begin{exemple}
	Faisons par exemple la division euclidienne de 377 par 12 :
	$$
		\begin{array}{rrrrr|r}
			                                        &   & 3                   & 7                         & 7                         & 12                                                             \\
			\cline{6-6}
			\textcolor{blue}{0} × 12 = 0 \to        & - & \textcolor{blue}{0} &                           &                           &                                                                \\
			\cline{2-5}
			                                        &   &                     &                           &                           & \textcolor{blue}{0}\textcolor{red}{3}\textcolor{OliveGreen}{1} \\
			                                        &   & 3                   & 7                         &                           &                                                                \\
			\textcolor{red}{3} × 12 = 36 \to        & - & \textcolor{red}{3}  & \textcolor{red}{6}        &                           &                                                                \\
			\cline{2-5}
			                                        &   &                     &                           &                           &                                                                \\
			                                        &   &                     & 1                         & 7                         &                                                                \\
			\textcolor{OliveGreen}{1} × 12 = 12 \to & - &                     & \textcolor{OliveGreen}{1} & \textcolor{OliveGreen}{2} &                                                                \\
			\cline{2-5}
			                                        &   &                     &                           &                           &                                                                \\
			                                        &   &                     &                           & 5                         &
		\end{array}
	$$

	On obtient ainsi un \textbf{quotient} de 31, et un \textbf{reste} de 5.
\end{exemple}

\begin{cours}
	Un \myuline{nombre premier} est un nombre qui n'a que 1 et lui même comme diviseurs.
\end{cours}

Note: il y a une \textbf{infinité} de nombres premiers.

\begin{exemple}
	2, 3, 5 et 7 sont des nombres premiers.
\end{exemple}

\newpage

On peut obtenir tous les nombres premiers entre 1 et 100 en utilisant un \myuline{crible d'Eratosthène} : \\

\textbf{Règles :}
\begin{itemize} \setlength\itemsep{0.3em}
	\item Barrer le nombre 1.
	\item Entourer le 2 (premier nombre non barré), puis barrer tous ses multiples.
	\item Entourer le premier nombre ni entouré ni barré, et barrer tous ses multiples.
	\item Répéter la consigne précédente, jusqu'à ce que tous les nombres soient soit entourés soit barrés.
\end{itemize}
\vspace{0.7cm}

\begin{tabular}{|c|c|c|c|c|c|c|c|c|c|}
	\hline
	\xcancel{1}  & \circled{\ 2} & \circled{\ 3} & \xcancel{4}  & \circled{\ 5} & \xcancel{6}  & \circled{\ 7} & \xcancel{8}  & \xcancel{9}  & \xcancel{10}  \\ \hline
	\circled{11} & \xcancel{12}  & \circled{13}  & \xcancel{14} & \xcancel{15}  & \xcancel{16} & \circled{17}  & \xcancel{18} & \circled{19} & \xcancel{20}  \\ \hline
	\xcancel{21} & \xcancel{22}  & \circled{23}  & \xcancel{24} & \xcancel{25}  & \xcancel{26} & \xcancel{27}  & \xcancel{28} & \circled{29} & \xcancel{30}  \\ \hline
	\circled{31} & \xcancel{32}  & \xcancel{33}  & \xcancel{34} & \xcancel{35}  & \xcancel{36} & \circled{37}  & \xcancel{38} & \xcancel{39} & \xcancel{40}  \\ \hline
	\circled{41} & \xcancel{42}  & \circled{43}  & \xcancel{44} & \xcancel{45}  & \xcancel{46} & \circled{47}  & \xcancel{48} & \xcancel{49} & \xcancel{50}  \\ \hline
	\xcancel{51} & \xcancel{52}  & \circled{53}  & \xcancel{54} & \xcancel{55}  & \xcancel{56} & \xcancel{57}  & \xcancel{58} & \circled{59} & \xcancel{60}  \\ \hline
	\circled{61} & \xcancel{62}  & \xcancel{63}  & \xcancel{64} & \xcancel{65}  & \xcancel{66} & \circled{67}  & \xcancel{68} & \xcancel{69} & \xcancel{70}  \\ \hline
	\circled{71} & \xcancel{72}  & \circled{73}  & \xcancel{74} & \xcancel{75}  & \xcancel{76} & \xcancel{77}  & \xcancel{78} & \circled{79} & \xcancel{80}  \\ \hline
	\xcancel{81} & \xcancel{82}  & \circled{83}  & \xcancel{84} & \xcancel{85}  & \xcancel{86} & \xcancel{87}  & \xcancel{88} & \circled{89} & \xcancel{90}  \\ \hline
	\xcancel{91} & \xcancel{92}  & \xcancel{93}  & \xcancel{94} & \xcancel{95}  & \xcancel{96} & \circled{97}  & \xcancel{98} & \xcancel{99} & \xcancel{100} \\ \hline
\end{tabular}

\begin{cours}[Décomposition en nombres premiers]
	Tout nombre peut être \myuline{décomposé} en un produit de nombres premiers.

	Pour trouver tous les diviseurs premiers d'un nombre, il faut essayer de diviser ce nombre par \textbf{tous} les nombres premiers qui lui sont inférieurs, jusqu'à n'avoir que des nombres premiers.
\end{cours}

\begin{exemple}
	\begin{itemize}
		\item On veut décomposer 15 en nombres premiers :
		      \begin{itemize}
			      \item $15$ n'est pas un multiple de $2$.
			      \item $15$ est un multiple de $3$: on écrit $15 = 3 × 5$.
			      \item $3$ et $5$ sont tous les deux premiers, donc on peut s'arrêter là.
		      \end{itemize}
		\item On veut décomposer 18 en nombres premiers :
		      \begin{itemize}
			      \item $18$ est un multiple de $2$: on écrit $18 = 2 × 9$.
			      \item $9$ n'est pas un multiple de $2$.
			      \item $9$ est un multiple de $3$: on écrit $18 = 2 × 3 × 3$.
			      \item $2$ et $3$ sont tous les deux premiers, donc on peut s'arrêter là.
		      \end{itemize}
		      On remarque que le même nombre premier peut apparaître \textbf{plusieurs fois} dans la décomposition !
		\item On veut décomposer $231$ en nombres premiers :
		      \begin{itemize}
			      \item Grâce aux \hyperref[cours:criteres-de-divisibilite]{Critères de divisibilité}, on peut déterminer que $231$ et un multiple de $3$: on écrit $231 = 3 × 77$.
			      \item $77$ n'est pas un multiple de $2$.
			      \item $77$ n'est pas un multiple de $3$.
			      \item $77$ n'est pas un multiple de $5$.
			      \item $77$ est un multiple de $7$: on écrit $231 = 3 × 7 × 11$.
			      \item $3$, $7$ et $11$ sont tous premiers, donc on peut s'arrêter là.
		      \end{itemize}
		\item $32 = 2 × 16 = 2 × 2 × 8 = 2 × 2 × 2 × 4 = 2 × 2 × 2 × 2 × 2$.
	\end{itemize}
\end{exemple}

\begin{cours}
	Le \myuline{PGCD} est le \myuline{Plus Grand Commun Diviseur} : c’est le plus grand nombre qui divise deux nombres donnés.

	Pour le calculer :
	\begin{itemize}
		\item On fait la liste des diviseurs premiers des deux nombres.
		\item On prend tous les nombres qui apparaissent dans les \textbf{deux} listes, et on les multiplient entre eux.
	\end{itemize}
\end{cours}

\begin{exemple}
	On veut calculer le PGCD de 12 et de 20 (noté PGCD(12, 20)) :
	\begin{align*}
		12 =  & \circled{2} × \circled{2} × 3 \\
		      &                               \\
		20  = & \circled{2} × \circled{2} × 5
	\end{align*}

	Donc PGCD(12, 20) = 2 × 2 = 4.
\end{exemple}

\begin{exemple}
	On veut calculer le PGCD de 60 et de 126 :
	\begin{align*}
		60 =   & \circled{2} × 2 × \circled{3} × 5 \\
		       &                                   \\
		126  = & \circled{2} × \circled{3} × 3 × 7
	\end{align*}

	Donc PGCD(60, 126) = 2 × 3 = 6.
\end{exemple}

% Bonus 5eme 5
% \begin{cours}
% 	Le \myuline{PPCM} est le \myuline{Plus Petit Commun Multiple}. C’est le plus petit nombre qui est un multiple de deux nombres donnés.

% 	Pour le trouver :
% 	\begin{itemize}
% 		\item On fait la liste des diviseurs premiers des deux nombres.
% 		\item Chaque fois qu’un nombre apparaît dans les deux listes, on l’enlève (des deux listes!) et on le met de côté.
% 		\item On multiplie les nombres mis de côté avec les nombres restants dans les listes.
% 	\end{itemize}
% \end{cours}

% \begin{exemple}
% 	On veut calculer le PPCM de 12 et de 30 :
% 	\begin{align*}
% 		12 =             & \cancel{2} × 2 × \cancel{3} \\
% 		                 &                             \\
% 		30 =             & \cancel{2} × \cancel{3} × 5 \\
% 		\text{de côté} = & 2, 3
% 	\end{align*}

% 	Donc PPCM(12, 30) = 2 × 3 × 2 × 5 = 60.
% \end{exemple}

\end{document}