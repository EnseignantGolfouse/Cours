\documentclass[a4paper]{article}

\usepackage[utf8]{inputenc}     % support des caractères spéciaux (é, þ, ...)
\usepackage{fontspec}           % Pour changer la police
\defaultfontfeatures{Scale=1.4} % text scale
\setmainfont[Ligatures=TeX]{FreeSerif} % Police avec support pour l'UTF-8
\usepackage{unicode-math}       % Symboles unicodes en mode math
\usepackage[francais]{babel}    % françisation partielle de l'output
\usepackage{natbib}             % bibliographie
\usepackage{amsthm}             % maths
\usepackage{amsmath}            % maths
\usepackage{mathtools}          % maths
\usepackage{graphicx}           % figures
\usepackage{titlesec}           % permet de changer l'affichage des sections et autres
\usepackage{color}              % couleurs rgb
\usepackage[tikz]{mdframed}     % boites fancy autour des théorèmes et autres
\usepackage{contour}            % Pour souligner
\usepackage{ulem}               % Pour souligner
\usetikzlibrary{shadows}        % effet d'ombre sur les boites ('shadow = true')
\usepackage{nopageno}           % pas de numéros de page
\usepackage[makeroom]{cancel}   % texte barré
\usepackage{hyperref}           % références
\usepackage{multicol}           % avoir le texte en plusieurs colonnes
\usepackage{caption}            % Custom image captions
\usepackage{floatrow}           % ??? plusieurs images côte à côte
\usepackage{pstricks-add}       % figures ( en l'occurence des droites graduées)
\usepackage[thinlines]{easytable}

% unité de longueur pour pstricks
\psset{unit=0.4cm}

% marges
\addtolength{\topmargin}{-2cm}
\addtolength{\textheight}{4cm}

% renommage de la table des matières
\addto\captionsfrench{\renewcommand*\contentsname{Nom de la table des matières}}
\setcounter{tocdepth}{1}

\newcommand{\somme}[2]{\overset{#2}{\underset{#1}{\sum}}}
\newcommand{\produit}[2]{\overset{#2}{\underset{#1}{\prod}}}
\newcommand{\sommedirecte}[2]{\overset{#2}{\underset{#1}{\oplus}}}
\renewcommand*{\proofname}{Preuve}

\renewcommand{\ULdepth}{1.8pt}
\contourlength{0.8pt}

% Commande de soulignage
\newcommand{\myuline}[1]{%
  \uline{\phantom{#1}}%
  \llap{\contour{white}{#1}}%
}

% Pour entourer un nombre
\newcommand*\circled[1]{\tikz[baseline=(char.base)]{
            \node[shape=circle,draw,inner sep=2pt] (char) {#1};}}


%%%%%%%%%%%%%%%%%%

\title{Crible d'Ératosthène}
\date{}

%%%%%%%%%%%%%%%%%%

\begin{document}

\begin{center}
	\begin{TAB}(r,0cm,0cm)[0pt,10cm,10cm]{|c|c|c|c|c|c|c|c|c|c|}{|c|c|c|c|c|c|c|c|c|c|}
		1 & 2 & 3 & 4  & 5 & 6  & 7 & 8  & 9  & 10  \\
		11          & 12  & 13  & 14 & 15  & 16 & 17  & 18 & 19 & 20  \\
		21          & 22  & 23  & 24 & 25  & 26 & 27  & 28 & 29 & 30  \\
		31          & 32  & 33  & 34 & 35  & 36 & 37  & 38 & 39 & 40  \\
		41          & 42  & 43  & 44 & 45  & 46 & 47  & 48 & 49 & 50  \\
		51          & 52  & 53  & 54 & 55  & 56 & 57  & 58 & 59 & 60  \\
		61          & 62  & 63  & 64 & 65  & 66 & 67  & 68 & 69 & 70  \\
		71          & 72  & 73  & 74 & 75  & 76 & 77  & 78 & 79 & 80  \\
		81          & 82  & 83  & 84 & 85  & 86 & 87  & 88 & 89 & 90  \\
		91          & 92  & 93  & 94 & 95  & 96 & 97  & 98 & 99 & 100 \\
	\end{TAB}
\end{center}

\textbf{Étapes :}
\begin{enumerate}
	\item Barrer le nombre 1.
	\item Entourer le nombre 2 (prochain nombre non barré), et barrer tous ses multiples.
	\item Entourer le prochain nombre ni barré ni entouré, et barrer tous ses multiples.
	\item Répéter la 3\textsuperscript{ème} étape, jusqu'à ce que tous les nombres soient barrés ou entourés.
\end{enumerate}


\begin{center}
	\begin{TAB}(r,0cm,0cm)[0pt,10cm,10cm]{|c|c|c|c|c|c|c|c|c|c|}{|c|c|c|c|c|c|c|c|c|c|}
		\xcancel{1}  & \circled{\ 2} & \ \circled{\ 3} & \xcancel{4}  & \ \circled{\ 5} & \xcancel{6}  & \ \circled{\ 7} & \xcancel{8}  & \xcancel{9}  & \xcancel{10}  \\
		\circled{11} & \xcancel{12}  & \circled{13}    & \xcancel{14} & \xcancel{15}    & \xcancel{16} & \circled{17}    & \xcancel{18} & \circled{19} & \xcancel{20}  \\
		\xcancel{21} & \xcancel{22}  & \circled{23}    & \xcancel{24} & \xcancel{25}    & \xcancel{26} & \xcancel{27}    & \xcancel{28} & \circled{29} & \xcancel{30}  \\
		\circled{31} & \xcancel{32}  & \xcancel{33}    & \xcancel{34} & \xcancel{35}    & \xcancel{36} & \circled{37}    & \xcancel{38} & \xcancel{39} & \xcancel{40}  \\
		\circled{41} & \xcancel{42}  & \circled{43}    & \xcancel{44} & \xcancel{45}    & \xcancel{46} & \circled{47}    & \xcancel{48} & \xcancel{49} & \xcancel{50}  \\
		\xcancel{51} & \xcancel{52}  & \circled{53}    & \xcancel{54} & \xcancel{55}    & \xcancel{56} & \xcancel{57}    & \xcancel{58} & \circled{59} & \xcancel{60}  \\
		\circled{61} & \xcancel{62}  & \xcancel{63}    & \xcancel{64} & \xcancel{65}    & \xcancel{66} & \circled{67}    & \xcancel{68} & \xcancel{69} & \xcancel{70}  \\
		\circled{71} & \xcancel{72}  & \circled{73}    & \xcancel{74} & \xcancel{75}    & \xcancel{76} & \xcancel{77}    & \xcancel{78} & \circled{79} & \xcancel{80}  \\
		\xcancel{81} & \xcancel{82}  & \circled{83}    & \xcancel{84} & \xcancel{85}    & \xcancel{86} & \xcancel{87}    & \xcancel{88} & \circled{89} & \xcancel{90}  \\
		\xcancel{91} & \xcancel{92}  & \xcancel{93}    & \xcancel{94} & \xcancel{95}    & \xcancel{96} & \circled{97}    & \xcancel{98} & \xcancel{99} & \xcancel{100} \\
	\end{TAB}
\end{center}

\end{document}