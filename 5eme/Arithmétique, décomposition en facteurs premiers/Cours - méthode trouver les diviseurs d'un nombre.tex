\documentclass[a4paper]{article}

\usepackage{../préambule}

%%%%%%%%%%%%%%%%%%

\title{Méthode : trouver les diviseurs premiers d'un nombre}
\date{}

%%%%%%%%%%%%%%%%%%


\begin{document}

\maketitle

\begin{methode}[Décomposer un nombre en nombres premiers]
	On dessine une colonne, dans laquelle on met le nombre à décomposer.

	Puis on cherche le \textit{plus petit diviseur premier} du nombre en bas à gauche, et on l'écrit à droite. On calcule le \textbf{quotient} de la division euclidienne de ce nombre par le diviseur trouvé, et on le place en bas de la colonne de gauche.

	Et ce jusqu'à ce qu'il n'y ai plus que 1 au bas de la colonne de gauche.

	Les nombres dans la colonne de droite forment alors la \textbf{décomposition en facteurs premiers} du nombre.
\end{methode}

\begin{exemple}
	On cherche à décomposer $21$ :
	\begin{itemize}
		\setlength\itemsep{0.4em}
		\item[∙] \begin{tabular}{c|c}
			      21 &
		      \end{tabular}
		\item[∙] \begin{tabular}{c|c}
			      21 & 3 \\
			      7  &
		      \end{tabular} (car $3 × 7 = 21$)
		\item[∙] \begin{tabular}{c|c}
			      21 & 3 \\
			      7  & 7 \\
			      1  &
		      \end{tabular}
	\end{itemize}

	On obtient ainsi $$ 21 = 3 × 7 $$
\end{exemple}

\begin{exemple}
	On cherche à décomposer $60$ :
	\begin{itemize}
		\setlength\itemsep{0.4em}
		\item[∙] \begin{tabular}{c|c}
			      60 &
		      \end{tabular}
		\item[∙] \begin{tabular}{c|c}
			      60 & 2 \\
			      30 &
		      \end{tabular} (car $2 × 30 = 60$)
		\item[∙] \begin{tabular}{c|c}
			      60 & 2 \\
			      30 & 2 \\
			      15 &
		      \end{tabular} (car $2 × 15 = 30$)
		\item[∙] \begin{tabular}{c|c}
			      60 & 2 \\
			      30 & 2 \\
			      15 & 3 \\
			      5  &
		      \end{tabular} (car $3 × 5 = 15$)
		\item[∙] \begin{tabular}{c|c}
			      60 & 2 \\
			      30 & 2 \\
			      15 & 3 \\
			      5  & 5 \\
			      1  &
		      \end{tabular}
	\end{itemize}

	On obtient ainsi $$ 60 = 2 × 2 × 3 × 5 $$
\end{exemple}

\end{document}