\documentclass[a4paper]{article}

\usepackage[utf8]{inputenc}     % support des caractères spéciaux (é, þ, ...)
\usepackage{fontspec}           % Pour changer la police
\defaultfontfeatures{Scale=1.4} % text scale
\setmainfont[Ligatures=TeX]{FreeSerif} % Police avec support pour l'UTF-8
\usepackage{unicode-math}       % Symboles unicodes en mode math
\usepackage[francais]{babel}    % françisation partielle de l'output
\usepackage{natbib}             % bibliographie
\usepackage{amsthm}             % maths
\usepackage{amsmath}            % maths
\usepackage{mathtools}          % maths
\usepackage{graphicx}           % figures
\usepackage{titlesec}           % permet de changer l'affichage des sections et autres
\usepackage{color}              % couleurs rgb
\usepackage[dvipsnames]{xcolor} % plus de couleurs
\usepackage[tikz]{mdframed}     % boites fancy autour des théorèmes et autres
\usepackage{contour} % Pour souligner
\usepackage{ulem}    % Pour souligner
\usetikzlibrary{shadows}        % effet d'ombre sur les boites ('shadow = true')
\usepackage{nopageno} % no page numbers

% renommage de la table des matières
\addto\captionsfrench{\renewcommand*\contentsname{Nom de la table des matières}}
\setcounter{tocdepth}{1}


% Définition des théorèmes, remarques...


\definecolor{theoremcolor}{RGB}{220, 20, 20}
\definecolor{definitioncolor}{RGB}{0, 170, 40}
\definecolor{remarkcolor}{RGB}{0, 0, 100}

\mdfdefinestyle{coursstyle}{
    linecolor=gray!60,
    linewidth=1pt,
    frametitlefontcolor=theoremcolor,
    frametitlebackgroundcolor=gray!20,
    frametitlerule=true,
    frametitlerulewidth=1pt,
    frametitlerulecolor=theoremcolor!50,
    frametitle={Cours},
    roundcorner=5pt,
    shadow=true,
    innertopmargin=\topskip,
}

\mdfdefinestyle{exemplestyle}{
    linecolor=black!70,
    linewidth=1pt,
    backgroundcolor=gray!0,
    frametitlefontcolor=remarkcolor,
    frametitlebackgroundcolor=gray!10,
    frametitlerulecolor=black!30,
    frametitle={Exemple},
    topline=false,
    bottomline=false,
    rightline=false,
    leftmargin=12,
    innertopmargin=\topskip,
}

\newmdenv[style=exemplestyle]{exemple}
\newmdenv[style=coursstyle]{cours}

% Titre optionel pour les blocks 'cours'
\makeatletter
\let\orig@cours=\cours
\def\cours{
  \@ifnextchar[{\cours@opt}{\orig@cours}
}
\def\cours@opt[#1]{
  \orig@cours[frametitle={Cours : #1}]
}
\makeatother

% environnement exercices
\newtheoremstyle{exostyle}
{\topsep}% espace avant
{1cm}% espace apres
{}% Police utilisee par le style de thm
{}% Indentation (vide = aucune, \parindent = indentation paragraphe)
{\bfseries}% Police du titre de thm
{.}% Signe de ponctuation apres le titre du thm
{ }% Espace apres le titre du thm (\newline = linebreak)
{\thmname{#1}\thmnumber{ #2}\thmnote{. \normalfont{\textit{#3}}}}% composants du titre du thm : \thmname = nom du thm, \thmnumber = numéro du thm, \thmnote = sous-titre du thm

\theoremstyle{exostyle}
\newtheorem{exercice}{Exercice} 

\newcommand{\somme}[2]{\overset{#2}{\underset{#1}{\sum}}}
\newcommand{\produit}[2]{\overset{#2}{\underset{#1}{\prod}}}
\newcommand{\sommedirecte}[2]{\overset{#2}{\underset{#1}{\oplus}}}
\renewcommand*{\proofname}{Preuve}

\renewcommand{\ULdepth}{1.8pt}
\contourlength{0.8pt}

% Commande de soulignage
\newcommand{\myuline}[1]{%
  \uline{\phantom{#1}}%
  \llap{\contour{white}{#1}}%
}

% Pour entourer un nombre
\newcommand*\circled[1]{\tikz[baseline=(char.base)]{
            \node[shape=circle,draw,inner sep=2pt] (char) {#1};}}


%%%%%%%%%%%%%%%%%%

\title{Exercices : Division euclidiennes, nombres premiers}
\date{10 septembre 2021}

%%%%%%%%%%%%%%%%%%


\begin{document}

\maketitle

\begin{exercice}
	Faire \textbf{sans} calculatrice les divisions euclidiennes suivantes :
	\begin{enumerate}
		\begin{minipage}{0.45\linewidth}
			\item $\begin{array}{r|r}
					423 & 5 \\
					\cline{2-2}
					    &
				\end{array}$
			\item $\begin{array}{r|r}
					2251 & 10 \\
					\cline{2-2}
					     &
				\end{array}$
			\item $\begin{array}{r|r}
					1174 & 11 \\
					\cline{2-2}
					     &
				\end{array}$
			\item $\begin{array}{r|r}
					3201 & 14 \\
					\cline{2-2}
					     &
				\end{array}$
		\end{minipage}
		\begin{minipage}{0.45\linewidth}
			\item $\begin{array}{r|r}
					745 & 8 \\
					\cline{2-2}
					    &
				\end{array}$
			\item $\begin{array}{r|r}
					8756 & 4 \\
					\cline{2-2}
					     &
				\end{array}$
			\item $\begin{array}{r|r}
					4985 & 23 \\
					\cline{2-2}
					     &
				\end{array}$
			\item $\begin{array}{r|r}
					874 & 17 \\
					\cline{2-2}
					    &
				\end{array}$
		\end{minipage}
	\end{enumerate}
\end{exercice}

\begin{exercice}
	Faire \textbf{avec} la calculatrice les divisions euclidiennes suivantes :
	\begin{enumerate}
		\begin{minipage}{0.45\linewidth}
			\item $\begin{array}{r|r}
					894523 & 452 \\
					\cline{2-2}
					       &
				\end{array}$
			\item $\begin{array}{r|r}
					54779612 & 54236 \\
					\cline{2-2}
					         &
				\end{array}$
		\end{minipage}
		\begin{minipage}{0.45\linewidth}
			\item $\begin{array}{r|r}
					4215637 & 3156 \\
					\cline{2-2}
					        &
				\end{array}$
			\item $\begin{array}{r|r}
					7530633 & 2013 \\
					\cline{2-2}
					        &
				\end{array}$
		\end{minipage}
	\end{enumerate}
\end{exercice}

\begin{exercice}
	Dire si les affirmations suivantes sont vraies ou fausses, et expliquer pourquoi. Ne pas utiliser la calculatrice.
	\begin{enumerate}
		\item $235$ est un multiple de $3$
		\item $712$ est un multiple de $5$
		\item $1022$ est un multiple de $2$
		\item $17$ est un diviseur de $1717$
		\item $11$ est un multiple de $1111$
		\item $5139$ est un multiple de $16$
	\end{enumerate}
\end{exercice}

\begin{exercice}
	\begin{itemize}
		\item Combien y-a-t'il de minutes dans une journée ? Et combien de secondes ?
		\item Alice a été en cours 420 minutes aujourd'hui. Combien d'heures a-t-elle passé en cours ?
		\item Bilal a lui compté les secondes ! Il en a compté $22320$. A-t-il passé un nombre rond d'heures en cours ? Si non, écrire le temps qu'il a passé en cours en heures et en minutes.
	\end{itemize}
\end{exercice}

\begin{exercice}
	\begin{enumerate}
		\item Lister tous les diviseurs de $140$.
		\item Lister tous les diviseurs de $42$.
		\item Lister tous les nombres qui sont diviseur de $140$ \textbf{et} de $42$.
	\end{enumerate}
\end{exercice}

\textbf{Si j'ai fini, je peux : \\}

\begin{itemize}
	\item Faire mes devoirs.
	\item Faire les exercices 54, 55, 56, 57 et 58 page 28.
\end{itemize}

\end{document}