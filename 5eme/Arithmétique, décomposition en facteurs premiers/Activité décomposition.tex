\documentclass[a4paper,12pt]{article}

\usepackage{préambule}

%%%%%%%%%%%%%%%%%%

\title{}
\date{14 septembre 2021}

%%%%%%%%%%%%%%%%%%


\begin{document}

\noindent\textbf{\huge \uline{Activité : Décomposition de nombres}}
\vspace{1em}

\makeatletter
\@date
\makeatother
\strut\newline

\begin{greybox}[frametitle={Consignes}]
	\begin{itemize}
		\item A chaque question, il y a une \textit{liste} de nombre, ainsi qu'une \textit{cible}.
		\item La cible est obtenue en \uline{multipliant} des nombres de la liste. Chaque nombre ne peut être utilisé qu'une seule fois.
		\item Il faut utiliser \uline{\textbf{le plus possible}} de nombres de la liste pour obtenir la cible.
	\end{itemize}
\end{greybox}

\textbf{Question 1} \\
\begin{itemize}
	\setlength\itemsep{0.3em}
	\item[\textit{Liste}]: $[2, 2, 5, 7, 11]$
	\item[\textit{Cible}]: $70 = ...$
\end{itemize}
\vspace{0.2cm}
Total de nombres utilisés: ...
\vspace{0.8cm}

\textbf{Question 2} \\
\begin{itemize}
	\setlength\itemsep{0.3em}
	\item[\textit{Liste}]: $[2, 3, 6, 7]$
	\item[\textit{Cible}]: $42 = ...$
\end{itemize}
\vspace{0.2cm}
Total de nombres utilisés: ...
\vspace{0.8cm}

\textbf{Question 3} \\
\begin{itemize}
	\setlength\itemsep{0.3em}
	\item[\textit{Liste}]: $[2, 3, 4, 5, 6, 7, 8, 9]$
	\item[\textit{Cible}]: $105 = ...$
\end{itemize}
\vspace{0.2cm}
Total de nombres utilisés: ...
\vspace{0.8cm}

\textbf{Question 4} \\
\begin{itemize}
	\setlength\itemsep{0.3em}
	\item[\textit{Liste}]: $[2, 2, 3, 4, 5, 7, 9, 11, 15]$
	\item[\textit{Cible}]: $660 = ...$
\end{itemize}
\vspace{0.2cm}
Total de nombres utilisés: ...
\vspace{0.8cm}

\textbf{Question 5} \\
\begin{itemize}
	\setlength\itemsep{0.3em}
	\item[\textit{Liste}]: $[2, 3, 3, 5, 9, 10, 11, 12, 13, 14]$
	\item[\textit{Cible}]: $1287 = ...$
\end{itemize}
\vspace{0.2cm}
Total de nombres utilisés: ...

\end{document}