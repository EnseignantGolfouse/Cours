\documentclass[a4paper]{article}

\usepackage{../../préambule}

\makeatletter
\renewcommand{\maketitle}{%
    \topskip0pt
	\@author \hfill \@date \\

	\begin{center}
		\begin{huge}
			\@title \\[2em]
		\end{huge}
	\end{center}
}
\makeatother

% marges
\addtolength{\oddsidemargin}{-1cm}
\addtolength{\evensidemargin}{-1cm}
\addtolength{\textwidth}{2cm}

\title{Contrôle n°1 : multiples, diviseurs, nombres premiers}
\date{22 septembre 2021}
\author{Nom : ..........}

\begin{document}

\maketitle

\begin{question}[(4 points)] Question de cours \\
	\begin{itemize}
		\setlength\itemsep{0.4em}
		\item[$\bullet$] L’écriture 42 = 7 × 6 permet de dire que 6 est un \dotfill de 42.
		\item[$\bullet$] Écris la liste des 4 premiers multiples de 14 : \dotfill
		\item[$\bullet$] Le nombre 39 est-il premier ? pourquoi ? \dotfill \\[0.5em] .\dotfill
		\item[$\bullet$] Un nombre est divisible par 3 si : \dotfill \\[0.5em] .\dotfill
	\end{itemize}
\end{question}

\begin{question}[(3,5 points)] QCM : entourer \myuline{la} ou \myuline{les} bonnes réponses.

	\renewcommand{\arraystretch}{1.8}
	\begin{tabular}[t]{|m{3.8cm}|c|c|c|}
		\firsthline
		$124$ est un multiple de :   & $4$                  & $3$                  & $2$                      \\ \hline
		$26$                         & a pour diviseur $13$ & a pour multiple $13$ & est un multiple de $13$  \\ \hline
		Parmi les égalités
		suivantes, donner la
		division euclidienne de
		$375$ par $14$               & $375=25×14 +25$      & $375=26×14+11$       & $375=27×14-3$            \\ \hline
		$21336$ est un multiple de : & $3$                  & $9$                  & $4$                      \\ \hline
		$17$ est :                   & Un diviseur de $34$  & Un multiple de $17$  & Le seul diviseur de $17$ \\ \hline
	\end{tabular}
	\renewcommand{\arraystretch}{1}
\end{question}

\begin{question}[(3 points)] Division euclidienne.
	\begin{enumerate}
		\item Faire la division euclidienne de 63 par 5 : \\[2cm]
		\item Faire la division euclidienne de 130 par 7 : \\[2cm]
		\item \renewcommand{\arraystretch}{1.3}
		      $
			      \begin{array}{r|r}
				      ??     & 14 \\
				      \cline{2-2}
				      \vdots & 18 \\
				      5      &
			      \end{array}
		      $ \renewcommand{\arraystretch}{1}

		      $??$ est égal à ....
	\end{enumerate}
\end{question}

\begin{question}[(2.5 points)]

	Un cuisinier a un lot de 85 crevettes.
	\begin{enumerate}
		\item Il voudrait les répartir dans 14 assiettes de manière que chaque assiette contienne le même nombre de crevettes et il voudrait utiliser toutes les crevettes. Est-ce possible ? Pourquoi ? \\[0.3cm]
		      .\dotfill
		\item S’il n’utilise pas toutes les crevettes, combien peut-il en mettre au maximum dans chaque assiette ? \\[0.3cm]
		      .\dotfill
	\end{enumerate}
\end{question}

\begin{question}[(2 points)]
	Dans une partie de cartes on doit répartir entre les joueurs $18$ jetons noirs et $12$ jetons blancs. Chaque joueur doit recevoir le même nombre de jetons noirs et le même nombre de jetons blancs.
	\begin{enumerate}
		\item Peut-il y avoir $2$ joueurs ?
		\item Peut-il y avoir $9$ joueurs ?
	\end{enumerate}
\end{question}

\begin{question}[(2,5 points)]
	Pour préparer la décoration d’une fête, Selma, Marion et Rachid possèdent $36$ ballons rouges et $45$ ballons noirs. Ils veulent faire des paquets en utilisant tous les ballons, de telle sorte que tous les paquets contiennent le même nombre de ballons noirs et le même nombre de ballons rouges. Selma dit qu’elle peut faire trois paquets. Marion dit qu’elle peut en faire quatre. Rachid leur
	annonce qu’il peut en faire cinq.

	Qui dit vrai, et pourquoi ? \\[0.5cm]
	.\dotfill \\[0.5cm]
	.\dotfill \\[0.5cm]
	.\dotfill \\[0.5cm]
	.\dotfill
\end{question}

\begin{question}[(2,5 points)]
	Léa a oublié le code à 4 chiffres de son entrée. Elle sait que :
	\begin{itemize}
		\setlength\itemsep{0.3em}
		\item[] Le chiffre des unités divise tous les nombres entiers.
		\item[] Le chiffre des dizaines multiplié par celui des milliers donne le chiffre des centaines.
		\item[] Le chiffre des milliers est impair.
		\item[] La somme des chiffres est 12 et les quatre chiffres sont différents.
	\end{itemize}

	Quel est le code ? \\ \\
	.\dotfill
\end{question}

\end{document}