\documentclass[a4paper,12pt]{article}

\usepackage{préambule}
\usepackage{xargs}
\usetikzlibrary{calc}

\newcommand{\exerciceEspacement}{0.6em}

% Arguments:
%   - (optionel, défaut = 1) taille des graduations
%   - Abscisse minimale
%   - Abscisse maximale
%   - (optionel, défaut = 0) hauteur de la droite 
\newcommandx{\DroiteGraduee}[4][1=1, 4=0]{
	\pgfmathtruncatemacro{\INTERNALDroiteGradueeLength}{#3 - #2}
	\pgfmathtruncatemacro{\INTERNALDroiteGradueeSecondStep}{#1 + #2}

	\draw[\myArrow] (-5.5,#4) -- ++(11, 0);

	\foreach \x in {#2,\INTERNALDroiteGradueeSecondStep,...,#3} {
		\draw
		(\x * 10 / \INTERNALDroiteGradueeLength - #2 * 10 / \INTERNALDroiteGradueeLength - 5,#4)
		-- ++(0,-0.2) node[below] {\x};
	}
}

\TitreDActivite{Activité : Distances}

\begin{document}

\maketitle

\begin{exercice}\

	\begin{center}
		\begin{tikzpicture}[scale=0.9]
			\coordinate (A) at (-4,0);
			\coordinate (B) at (-1,0);
			\coordinate (C) at (3,0);

			\DroiteGraduee{-5}{5}

			\foreach \p in {A,B,C} {
					\node at (\p) {×};
					\node[above] at (\p) {$\p$};
				}
		\end{tikzpicture}
	\end{center}

	\begin{enumerate}
		\item Quelle est la distance entre $A$ et $B$ ? ........
		\item Quelle est la distance entre $B$ et $C$ ? ........
		\item Quelle est la distance entre $A$ et $C$ ? ........
		\item On place un point $D$ d'abscisse $20$. Quelle est la distance entre $B$ et $D$ ? ........
		\item On place un point $E$ d'abscisse $200$. Quelle est la distance entre $C$ et $E$ ? ........
	\end{enumerate}
\end{exercice}

\vspace{\exerciceEspacement}

\begin{exercice}\

	Répond \textbf{sans} dessiner de droite !

	On place trois points sur une droite :
	\begin{itemize}
		\item $A$ d'abscisse $24$.
		\item $B$ d'abscisse $-8$.
		\item $C$ d'abscisse $15$.
	\end{itemize}
	\begin{enumerate}
		\item Quelle est la distance entre $A$ et $B$ ? ........
		\item      Quelle est la distance entre $B$ et $C$ ? ........
		\item      Quelle est la distance entre $A$ et $C$ ? ........
	\end{enumerate}
\end{exercice}

\vspace{\exerciceEspacement}

\begin{exercice}\

	On place sur une droite :
	\begin{itemize}
		\item Un point $A$ d'abscisse $8$.
		\item Un point $B$, à $50$ unités à \uline{droite} de $A$.
		\item Un point $C$, à $15$ unités à \uline{gauche} de $B$.
		\item Un point $D$, à $25$ unités à \uline{gauche} de $C$.
	\end{itemize}


	Quelle est la distance entre $A$ et $D$ ? ........
\end{exercice}

\vspace{\exerciceEspacement}

\begin{exercice}\

	Une personne se promène le long d'une droite, et effectue une série de mouvement très précise :

	\begin{itemize}
		\item Tout d'abord, elle se place en $0$.
		\item Elle \textbf{avance} de $80$ pas.
		\item Elle \textbf{recule} de $160$ pas.
		\item Elle \textbf{recule} de $120$ pas.
		\item Elle \textbf{avance} de $240$ pas.
		\item Puis, \uline{elle se retourne}.
		\item Elle \textbf{avance} de $40$ pas.
		\item Elle \textbf{recule} de $80$ pas.
	\end{itemize}

	Écrire une expression qui correspond à tous ces mouvements :
	\vspace{0.4em}
	\begin{center}
		............................
	\end{center}
	\vspace{0.4em}

	Quelle est alors sa position d'arrivée ? .........
\end{exercice}

\end{document}