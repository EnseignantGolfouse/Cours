% !TEX root = ./Cours.tex
\documentclass[../€Cours-complet/Cours-complet]{subfiles}

\newcommand{\myarrow}{-{Latex[length=1mm, width=1.5mm]}}

\titleorchapter{Nombres relatifs (partie 2)}{11}

\begin{document}

\maketitleCours

\section{Addition}

\begin{greybox}[frametitle={Rappel}]
	La \textbf{distance à zéro} d'un nombre relatif est le nombre \uline{positif} situé après le signe.
\end{greybox}

\begin{exemple}
	La distance à zéro de $+5$ est $5$.

	La distance à zéro de $-2$ est $2$.
\end{exemple}

\begin{cours}[Addition de nombres relatifs, cas 1]
	Si deux nombres relatifs ont \uline{le même signe}, alors leur somme a :
	\begin{itemize}
		\item Le même signe que ces deux nombres.
		\item Pour distance à zéro, la \textbf{somme} de leur distances à zéro.
	\end{itemize}
\end{cours}

\begin{exemple}
	\begin{itemize}
		\item Pour calculer $5,4 + 3,5$ :
		      \begin{itemize}
			      \item Les deux nombres sont positifs, donc la somme est positive.
			      \item La somme de leurs distances à zéro est $5,4 + 3,5 = 8,9$.
		      \end{itemize}
		      Donc la somme de $5,4$ et $3,5$ est $+8,9$ (ou juste $8,9$).
		\item Pour calculer $-2 + (-7)$ :
		      \begin{itemize}
			      \item Les deux nombres sont négatifs, donc la somme est négative.
			      \item La somme de leurs distances à zéro est $2 + 7 = 9$.
		      \end{itemize}
		      Donc la somme de $-2$ et $-7$ est $-9$.
	\end{itemize}
\end{exemple}

\begin{cours}[Addition de nombres relatifs, cas 2]
	Si deux nombres relatifs sont de signes \uline{contraires}, alors leur somme a
	\begin{itemize}
		\item Le signe du nombre qui a la plus grande distance à zéro.
		\item Pour distance à zéro, la \textbf{différence} de leurs distances à zéro.
	\end{itemize}
\end{cours}

\begin{exemple}
	Pour calculer $-8 + 6$ :
	\begin{itemize}
		\item Le nombre qui à la plus grande distance à zéro est $-8$, donc le résultat est \textit{négatif}.
		\item La différence de leurs distances à zéro est $8 - 6 = 2$.
	\end{itemize}
	Donc la somme de $-8$ et de $6$ est $-2$.
\end{exemple}

\section{Nombres opposés}

\begin{cours}
	Deux nombres sont \textbf{opposés} si leur somme est égale à zéro.

	De manière équivalente, deux nombres opposés :
	\begin{itemize}
		\item Sont de signes contraires.
		\item Ont la même distance à zéro.
	\end{itemize}
\end{cours}

\begin{exemple}
	\begin{itemize}
		\item $3,2$ et $-3,2$ sont opposés.
		\item L'opposé de $-4,6$ est $+4,6$ (ou seulement $4,6$).
	\end{itemize}
\end{exemple}

\section{Soustraction}

\begin{cours}
	Pour \textbf{soustraire} un nombre relatif, on ajoute son opposé.
\end{cours}

\begin{exemple}
	\begin{minipage}{0.45\textwidth}
		\begin{align*}
			A & = -5 - 2    \\
			  & = -5 + (-2) \\
			  & = -(5 + 2)  \\
			  & = -7
		\end{align*}
	\end{minipage}
	\begin{minipage}{0.45\textwidth}
		\begin{align*}
			B & = 3 - (-8,7) \\
			  & = 3 + 8,7    \\
			  & = 11,7
		\end{align*}
	\end{minipage}
\end{exemple}

\begin{cours}
	Sue une droite graduée, la \textbf{distance} entre deux point est égale à \uline{la différence} entre la plus grande abscisse et la plus petite.
\end{cours}

\begin{exemple}
	\begin{center}
		\begin{tikzpicture}[scale=1.5]
			\coordinate (A) at (-2,0);
			\coordinate (B) at (3.5,0);
			\draw[\myarrow] (-2.2,0) -- (4.5,0);
			\foreach \p in {-1,...,4} {
					\draw (\p,0) -- (\p,-0.2) node[below] {$\p$};
				}
			\draw (A) -- ++(0,-0.2);
			\foreach \p/\value in {A/-2,B/3{,}5} {
			\node at (\p) {\color{red}×};
			\node[above] at ($(\p) + (0,0.2)$) {\color{red}\p};
			\node[below] at ($(\p) - (0,0.2)$) {\color{red}$\value$};
			}
		\end{tikzpicture}
	\end{center}

	La distance entre {\color{red}A} et {\color{red}B} est égale à :

	$AB = 3,5 - (-2)$

	$AB = 3,5 + 2$

	$AB = 5,5$
\end{exemple}

\end{document}