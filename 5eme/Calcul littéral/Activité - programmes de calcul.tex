\documentclass[a4paper,12pt,landscape,twocolumn]{article}

\usepackage{préambule}
\usepackage{clipboard}

\setmathfont{Latin Modern Math}[version=lm]

\newcommand{\computermodern}[1]{{\setmainfont{Latin Modern Roman}\mathversion{lm}\large#1}}

\geometry{left=1cm}
\setlength{\columnsep}{0.8cm}

\TitreDActivite{Activité : Programmes de calcul}

\begin{document}

\Copy{Contenu}{
	\maketitle

	Voici un programme de calcul :

	\begin{greybox}
		\texttt{Choisir un nombre.}

		\texttt{Ajouter 5.}

		\texttt{Multiplier le résultat obtenu par 3.}

		\texttt{Soustraire 8 au résultat obtenu.}
	\end{greybox}

	\begin{exercice}\

		\begin{enumerate}
			\item Quel est le résultat si le nombre choisi est 6 ? \dotfill % 25
			\item Quel est le résultat si le nombre choisi est 11 ? \dotfill % 40
		\end{enumerate}
	\end{exercice}

	\begin{exercice}\

		Parmi les expressions suivantes, laquelle correspond au programme ?

		\setlength{\tabcolsep}{1.8em}
		\renewcommand{\arraystretch}{1.5}
		\begin{tabular}{ll}
			1. \computermodern{$x$} + 5 × 3 − 8   & 2. (\computermodern{$x$} + 5) × 3 − 8 \\
			3. \computermodern{$x$} + (5 × 3 − 8) & 4. (\computermodern{$x$} − 8) × 3 + 5
		\end{tabular}
	\end{exercice}
}

\newpage

\Paste{Contenu}

\end{document}