\documentclass[a4paper,12pt]{beamer}

\usepackage{préambule}
\usetikzlibrary{calc,arrows.meta}

\newcommand{\mysize}{\scriptsize}

\newcommand{\myarrow}{{Latex[length=1mm, width=1mm]}-{Latex[length=1mm, width=1mm]}}

\begin{document}

\begin{frame}
	\frametitle{Exercice à faire}

	\begin{center}
		\begin{tikzpicture}
			\coordinate (A) at (0,0);
			\coordinate (B) at (1,0);
			\coordinate (C) at (2,0);
			\coordinate (D) at (3,0);
			\coordinate (E) at (6,0);
			\coordinate (F) at (9,0);

			\draw[|-|] (A) -- node {\mysize /} (B);
			\draw[|-|] (B) -- node {\mysize /} (C);
			\draw[|-|] (C) -- node {\mysize /} (D);
			\draw[|-|] (D) -- node {\mysize //} (E);
			\draw[|-|] (E) -- node {\mysize //} (F);

			\draw[\myarrow] ($(A) - (0,0.5)$) -- node[below] {$x$} ($(B) - (0,0.5)$);

			\draw[\myarrow] ($(D) - (0,0.5)$) -- node[below] {$2$} ($(E) - (0,0.5)$);
		\end{tikzpicture}
	\end{center}

	Écrire la longueur totale du segment avec une expression.

	\begin{center}
		\begin{tikzpicture}
			\coordinate (A) at (0,0);
			\coordinate (B) at (2,0);
			\coordinate (C) at (4,0);
			\coordinate (D) at (5,0);
			\coordinate (E) at (7,0);
			\coordinate (F) at (8,0);
			\coordinate (G) at (10,0);

			\draw[|-|] (A) -- node {\mysize /} (B);
			\draw[|-|] (B) -- node {\mysize /} (C);
			\draw[|-|] (C) -- node {\mysize //} (D);
			\draw[|-|] (D) -- node {\mysize /} (E);
			\draw[|-|] (E) -- node {\mysize //} (F);
			\draw[|-|] (F) -- node {\mysize /} (G);

			\draw[\myarrow] ($(A) - (0,0.5)$) -- node[below] {$y$} ($(C) - (0,0.5)$);

			\draw[\myarrow] ($(C) - (0,0.5)$) -- node[below] {$1{,}5$} ($(D) - (0,0.5)$);
		\end{tikzpicture}
	\end{center}

	Écrire la longueur totale du segment avec une expression.
\end{frame}

\end{document}