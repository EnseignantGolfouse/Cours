\documentclass[a4paper,12pt]{article}

\usepackage{préambule}
\usepackage{clipboard}

\setmathfont[range=it]{Tex Gyre Schola Math}

\TitreDActivite{Activité : Abonnement}

\begin{document}

\maketitle

\begin{enonce}
	Ivan et Kevin vont souvent regarder des films ensemble.

	\begin{itemize}
		\item Ivan paie à chaque fois sa place à plein tarif : 7€ la place.
		\item Kevin, lui, a un abonnement : il l'a acheté 20€, mais il paie chaque place 4,50€.
	\end{itemize}
\end{enonce}

\begin{enumerate}
	\item Si ils vont voir 3 films :
	      \begin{enumerate}
		      \item Combien paie Ivan ? \dotfill % 21€
		      \item Combien paie Kevin ? \dotfill % 33,50€
		      \item Lequel paie le moins ? \dotfill % Ivan
	      \end{enumerate}
	\item Si ils vont voir 10 films :
	      \begin{enumerate}
		      \item Combien paie Ivan ? \dotfill % 70€
		      \item Combien paie Kevin ? \dotfill % 65€
		      \item Lequel paie le moins ? \dotfill % Kevin
	      \end{enumerate}
	\item On appelle $n$ le nombre de films qu'ils vont voir.
	      \begin{enumerate}
		      \item Écrire une expression pour le prix que paie Ivan. \dotfill
		      \item Écrire une expression pour le prix que paie Kevin. \dotfill
	      \end{enumerate}
	\item Remplir le tableau suivant :

	      \begin{center}
		      \renewcommand{\arraystretch}{1.5}
		      \begin{tabular}{|c|c|c|}
			      \hline
			      Nombre de places & Prix pour Ivan & Prix pour Kevin
			      \\ \hline
			      3                &                &
			      \\ \hline
			      4                &                &
			      \\ \hline
			      5                &                &
			      \\ \hline
			      6                &                &
			      \\ \hline
			      7                &                &
			      \\ \hline
			      8                &                &
			      \\ \hline
			      9                &                &
			      \\ \hline
			      10               &                &
			      \\ \hline
		      \end{tabular}
	      \end{center}

	      Y-a-t'il un nombre de places, pour lequel les deux paie le même prix ? \dotfill
	\item Les expressions $7n$ et $20 + 4{,}5n$ sont elles toujours égales ?
	\item Les expressions $7n$ et $20 + 4{,}5n$ sont elles égales pour un $n$ particulier ?
\end{enumerate}

\end{document}