\documentclass[a4paper,12pt,landscape,twocolumn]{article}

\usepackage{préambule}

\geometry{left=1cm}
\setlength{\columnsep}{0.8cm}

\TitreDActivite{Activité : Introduction}

\begin{document}

\newcommand{\contenu}{
	\maketitle

	Gwladys a noté l’exercice suivant :

	\begin{greybox}
		Calcule : \vspace{0.5em}

		\setlength{\tabcolsep}{15pt}
		\begin{tabular}{lll}
			$250 × 2 + 3 : $  & $250 × 3 + 3 : $  & $250 × 4 + 3 : $  \\
			$250 × 5 + 3 : $  & $25 × 6 + 3 : $   & $250 × 7 + 3 : $  \\
			$250 × 8 + 3 : $  & $250 × 9 + 3 : $  & $25 × 10 + 3 : $  \\
			$250 × 11 + 3 : $ & $250 × 12 + 3 : $ & $250 × 13 + 3 : $
		\end{tabular}
	\end{greybox}

	Gwladys veut téléphoner à Éric pour lui dicter l’exercice, mais il ne lui reste que quelques secondes de forfait. Elle \textbf{\uline{ne peut donc pas dicter tous les calculs}}. \vspace{1em}

	Quelle consigne, la plus courte possible, donner à Éric pour qu’il sache \textbf{\uline{exactement ce qu’il doit faire ?}}
}

\contenu
\newpage
\contenu

\end{document}