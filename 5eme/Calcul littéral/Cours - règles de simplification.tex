\documentclass[a4paper,12pt]{beamer}

\usepackage{préambule}

\begin{document}

\begin{frame}
	\frametitle{Règles de simplification}

	\begin{itemize}
		\item Entre un nombre et une lettre :

		      $3 × a = 3a$

		      $20 − 5 × x = 20 − 5 x$
		\item Entre 2 lettres :
		\item
		      $x × y = xy$

		      $2 × a × b = 2 ab$
		\item Entre un nombre et une parenthèse :

		      $2 × (x + 3) = 2(x + 3)$

		      $5 × (3 × x - 1) = 5(3x − 1)$
		\item Entre une lettre et une parenthèse :

		      $x × (7+ y) = x (7+ y)$

		      $6 × x × (2 + y)= 6x(2 + y)$
		\item Entre 2 parenthèses :

		      $(5 + x) × (3−2 y) = (5 + x) (3−2 y)$
		\item $1x$ s'écrit simplement $x$
	\end{itemize}
\end{frame}

\end{document}