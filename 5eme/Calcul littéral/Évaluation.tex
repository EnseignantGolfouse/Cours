\documentclass[a4paper,12pt,landscape,twocolumn]{article}

\usepackage{préambule}
\usepackage{clipboard}

\setmathfont[range=it]{Tex Gyre Schola Math}

\geometry{left=1cm}
\setlength{\columnsep}{0.8cm}

\TitreDEvaluation{Évaluation calcul littéral}
\author{}
\date{15 avril}

\makeatletter
\newcommand{\mydate}{
\@date
}
\makeatother

\begin{document}

\Copy{Contenu}{
	\maketitle

	\begin{exercice}\

		$x$ désigne un nombre quelconque. Écrire une expression littérale qui représente :
		\begin{enumerate}[label=\alph*.]
			\item La somme de $x$ et 3.
			\item La différence entre 12 et le produit de $x$ et 2.
		\end{enumerate}
	\end{exercice}

	\begin{exercice}\

		Calculer la valeur de l'expression $4x + 3$ pour :
		\begin{enumerate}[label=\alph*.]
			\item $x = 10$  % 43
			\item $x = 5$   % 23
			\item $x = 2{,}5$ % 13
		\end{enumerate}
	\end{exercice}

	\begin{exercice}\

		Calculer la valeur de l'expression $5x + 6 - 3x$ pour :
		\begin{enumerate}[label=\alph*.]
			\item $x = 6$  % 18
			\item $x = 12$ % 30
			\item $x = 3$  % 12
		\end{enumerate}
	\end{exercice}

	\begin{exercice}\

		Les expressions suivantes sont-elles des égalités ?
		\begin{enumerate}[label=\alph*.]
			\item $3x + 2 = 8$
			\item $x + 6 - 1$
			\item $8x × 2 = 16$
		\end{enumerate}
	\end{exercice}
}

\newpage

\setcounter{exercice}{0}
\Paste{Contenu}

\end{document}