\documentclass[a4paper,oneside,notitlepage,11pt]{book}

\usepackage{préambule}
\usetikzlibrary{angles,arrows,arrows.meta,calc,intersections,quotes,positioning}

\renewcommand{\thesection}{\arabic{section}}

\newmdenv[style=importantstyle]{myimportantboxinternal}
\newcommand{\myimportantbox}[3][]{
	\begin{myimportantboxinternal}[
		frametitlebackgroundcolor=#2!70,
		frametitlerulecolor=#2,
		#1]
		#3
	\end{myimportantboxinternal}
}

\makeatletter
\newcommand{\titleorchapter}[2]{
	\ifthenelse{\inmaindocument=1}{
		\newcommand{\MyCurrentChapterName}{#1}
	}{
		\title{Chapitre #2 : #1}
		\date{}
		\author{}
	}
}
\newcounter{maketitleAlready}
\setcounter{maketitleAlready}{0}
\newcommand{\maketitleCours}{
	\ifthenelse{\inmaindocument=1}{
		\chapter{\MyCurrentChapterName}
	}{\maketitle}
}
\makeatother

\newcommand{\inmaindocument}{0}

\title{Cours de Mathématiques}
\date{2021-2022}
\author{5ème}

\usepackage{subfiles} % inclure des fichiers auxiliaires

\begin{document}

\renewcommand{\inmaindocument}{1}

\maketitle

\subfile{../Arithmétique, décomposition en facteurs premiers/Cours}
\newpage

\subfile{../Priorités opératoires/Cours}
\newpage

\subfile{../Symétrie/Cours}
\newpage

\subfile{../Proportionnalité et tableaux, durées et horaires/Cours}
\newpage

\subfile{../Nombres relatifs (partie 1)/Cours}
\newpage

\subfile{../Angles, angles dans un triangle/Cours}
\newpage

\subfile{../Fraction : comparer et prendre la fraction d'un nombre/Cours}
\newpage

\subfile{../Triangles et cercles : inégalité triangulaire, médiatrice, hauteurs/Cours}
\newpage

\subfile{../Calcul littéral/Cours}
\newpage

\subfile{../Série de données/Cours}
\newpage

\subfile{../Nombres relatifs (partie 2)/Cours}
\newpage

\subfile{../Solides de l'espace, volume/Cours}

\end{document}