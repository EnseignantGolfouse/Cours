\documentclass[a4paper,11pt]{article}

\usepackage{../préambule}
\usepackage{clipboard}

\title{Chapitre 3 : Symétrie}
\date{}
\author{}

\begin{luacode}
dofile "symetrie.lua"
function draw_dashed(p1, p2)
	tex.print("\\draw[dashed] (", p1.x, ",", p1.y, ") -- (", p2.x, ",", p2.y,");")
end
\end{luacode}

\begin{document}

\Copy{Exemples}{
	\begin{exemple}
		\begin{tikzpicture}
			\directlua{
			local centre = { x = 5, y = 3 }
			local points = {
			{x = 2, y = 5},
			{x = 2, y = 4},
			{x = 1, y = 4},
			{x = 1.5, y = 3.5},
			{x = 2.5, y = 3.5},
			{x = 3, y = 4},
			{x = 2, y = 4},
			{x = 2, y = 4.2},
			{x = 1.2, y = 4.2},
			{x = 2, y = 5},
			}

			local points_symetrique = symetrie_centrale(centre, points)

			tikz_draw_points(points)
			tikz_draw_points(points_symetrique)
			draw_dashed(points[5], points_symetrique[5])
			draw_dashed(points[6], points_symetrique[6])
			draw_center(centre, "south west")
			}

			\draw[black,thick] (3.8,3.75) -- ++(0.1,-0.3);
			\draw[black,thick] (6,2.6) -- ++(0.1,-0.3);

			\draw[black,thick] (3.6,3.45) -- ++(0.1,-0.3);
			\draw[black,thick] (3.7,3.44) -- ++(0.1,-0.3);
			\draw[black,thick] (6.2,2.9) -- ++(0.1,-0.3);
			\draw[black,thick] (6.3,2.89) -- ++(0.1,-0.3);
		\end{tikzpicture}
	\end{exemple}
}

\Paste{Exemples}

\Paste{Exemples}

\Paste{Exemples}

\end{document}