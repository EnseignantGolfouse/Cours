\documentclass[a4paper,12pt,landscape]{article}

\usepackage{préambule}

\addtolength{\oddsidemargin}{-2cm}
\addtolength{\evensidemargin}{-2cm}
\addtolength{\textwidth}{4cm}

\title{Symétrie axiale et frises}
\date{}
\author{}

\makeatletter
\newcommand{\mymaketitle}{%
	{\tiny colle dans ton cahier d'exercices}
	
	\begin{center}
		\LARGE
		\myuline{\@title}
		\vspace{1em}
	\end{center}
}
\makeatother

\begin{document}

\mymaketitle

\begin{enumerate}
	\item Fait le symétrique de la figure par rapport à l'axe $(d₁)$.
	\item Fait le symétrique \textbf{de la figure obtenue} par rapport à l'axe $(d₂)$.

	      \begin{tikzpicture}
		      \draw[ultra thin,gray] (0,0) grid ++(18,6);
		      \draw[thick] (6,-0.2) -- ++(0,6.4) node[anchor=south] {$(d₁)$};
		      \draw[thick] (12,-0.2) -- ++(0,6.4) node[anchor=south] {$(d₂)$};

		      \draw[ultra thick] (1,1) -- ++(3,0) -- ++(0,0.25) -- ++(-0.5,0.25) -- ++(-0.5,1.5) -- ++(0.5,0.5) -- ++(-0.25,1)
		      % oreilles
		      ++(-0.25,0) -- ++(-0.25,0.5) -- ++(-0.25,-0.5) -- ++(-0.25,0.5) -- ++(-0.25,-0.5) ++(1.25,0)
		      -- ++(-1.5,0) -- ++(0,-0.5) -- ++(0.25,-1) -- ++(-1,-2) -- ++(-0.5,2);
	      \end{tikzpicture}
	\item Que voit-on à propos des figures dans les zones 1 et 3 ? \vspace{3em}
	\item Que ce passerait-il si on rajoutait d'autres zones/axes de symétries ? \vspace{6em}

	      Peut-on savoir (sans dessiner !) quelle figure la zone 254 contiendra-elle ? \vspace{6em}
\end{enumerate}

\end{document}