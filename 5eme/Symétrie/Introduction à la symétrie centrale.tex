\documentclass[a4paper,12pt]{article}

\usepackage{préambule}

\addtolength{\oddsidemargin}{-0.5cm}
\addtolength{\evensidemargin}{-0.5cm}
\addtolength{\textwidth}{1cm}

\title{Introduction à la symétrie centrale}
\date{}
\author{}

\makeatletter
\renewcommand{\maketitle}{%
{\tiny colle dans ton cahier d'exercices}

	\begin{center}
		\LARGE
		\myuline{\@title}
		\vspace{1em}
	\end{center}
}
\makeatother

\begin{document}

\maketitle

\begin{attention}[frametitle={⚠ Attention ⚠}]
	\center
	Dessine au \textbf{crayon à papier} !
\end{attention}

On va faire le symétrique d'un poisson.
\begin{enumerate}
	\item Fait le symétrique du poisson selon l'axe $(d₁)$.
	\item Fait le symétrique du \textbf{deuxième} poisson selon l'axe $(d₂)$.

	      \begin{tikzpicture}[scale=1,c/.style={insert path={circle[radius=1.5pt]}}]
		      \draw[step=1.0,gray,ultra thin] (0,0) grid (10,10);
		      \draw[black,thick] (5,-0.2) -- (5,10.2) node[anchor=south]{$(d₁)$};
		      \draw[black,thick] (-0.2,5) -- (10.2,5) node[anchor=west]{$(d₂)$};

		      \filldraw[black] (5,5) node[anchor=north west] {$O$} circle (2pt);

		      \draw[black,ultra thick]
		      (8,6) [c] node[anchor=south] {A}
		      -- ++(-1,2) [c] node[anchor=south west] {B}
		      -- ++(-1,0) [c] node[anchor=south] {C}
		      -- ++(1,1) [c] node[anchor=south] {D}
		      -- ++(0,-1)
		      -- ++(2,-1) [c] node[anchor=south] {E}
		      -- ++(0,-1) [c] node[anchor=west] {F}
		      -- ++(-1,0)
		      -- ++(1,1);

		      %   \draw[black,ultra thick] (4,8) -- ++(-1,1) -- ++(0,-1) -- ++(-2,-1) -- ++ (0,-1) -- ++ (1,0) -- ++(1,2) -- cycle;

		      %   \draw[black,ultra thick] (4,2) -- ++(-1,-1) -- ++(0,1) -- ++(-2,1) -- ++ (0,1) -- ++ (1,0) -- ++(1,-2) -- cycle;

		      %   \draw[color=red] (9,6) -- (1,4);
	      \end{tikzpicture}
	\item Peut-on passer en un seul coup du premier au troisième poisson ? \vspace{2em}
	\item Trace une ligne qui relie le nez du premier poisson (le point F) au nez du troisième. Quelle position occupe le point $O$ sur cette ligne ? \vspace{2em}

	      Vérifie que c'est également vrai pour les points A, B et C.
\end{enumerate}

\begin{attention}[frametitle={}]
	On dit que les deux poissons sont \textbf{symétriques par rapport au \myuline{point} $O$}.
\end{attention}

\end{document}