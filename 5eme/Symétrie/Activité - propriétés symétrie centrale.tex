\documentclass[a4paper,12pt]{article}

\usepackage{préambule}

\usetikzlibrary{calc,positioning}

\addtolength{\oddsidemargin}{-0.5cm}
\addtolength{\evensidemargin}{-0.5cm}
\addtolength{\textwidth}{1cm}

\makeatletter
\renewcommand{\maketitle}{%
{\scriptsize colle dans ton cahier d'exercices, et écrit sur la feuille}

	\begin{center}
		\LARGE
		\myuline{\@title}
		\vspace{1em}
	\end{center}
}
\makeatother

\title{Exercices symétrie centrale}
\date{}
\author{}

\begin{document}

\maketitle
\begin{attention}
	Fais les constructions en utilisant une règle et un compas.
\end{attention}

\begin{exercice}
	\begin{enumerate}
		\item Trace le symétrique de A et de B par rapport à O.
		\item Trace le symétrique de la droite $(d)$ par rapport au point O.
		\item Que peux-tu dire de ces deux droites ? \vspace{2em}
	\end{enumerate}

	\begin{tikzpicture}
		\coordinate (O) at (0,0);
		\coordinate (A) at (-1,1);
		\coordinate (B) at (2,2);
		\coordinate[rotate around={180:(O)}] (A') at (A);
		\coordinate[rotate around={180:(O)}] (B') at (B);

		\node[right = 0.2cm of O] {O};
		\node[above left = 0.2cm of A] {A};
		\node[above left = 0.2cm of B] {B};
		\draw[fill] (O) circle (2pt);
		\draw[fill] (A) circle (2pt);
		\draw[fill] (B) circle (2pt);
		\draw ($(A) + (A) - (B)$) -- ($(B) + (B) - (A)$) node[anchor=south] {$(d)$};

		% \node[above left = 0.2cm of A'] {A'};
		% \node[above left = 0.2cm of B'] {B'};
		% \draw[fill] (A') circle (2pt);
		% \draw[fill] (B') circle (2pt);
		\draw[opacity=0] ($(A') + (A') - (B')$) -- ($(B') + (B') - (A')$) node[anchor=south] {$(d')$};
	\end{tikzpicture}
\end{exercice}

\begin{exercice}
	\begin{enumerate}
		\item Construit C et D les symétriques de A et B par rapport à I.
		\item Construit E et F les symétriques de A et B par rapport à J.
		\item Trace les \textbf{segments} [AB], [CD] et [EF].
		\item Que peux-tu dire de ces trois segments ? \vspace{2em}
	\end{enumerate}
	\begin{tikzpicture}
		\coordinate (A) at (-1,1);
		\coordinate (B) at (2,2);
		\coordinate (I) at (0,0);
		\coordinate (J) at (3,-1);
		\coordinate[rotate around={180:(I)}] (C) at (A);
		\coordinate[rotate around={180:(I)}] (D) at (B);
		\coordinate[rotate around={180:(J)}] (E) at (A);
		\coordinate[rotate around={180:(J)}] (F) at (B);

		\foreach \P in {A,B,I,J} {
				\node[above = 0.2cm of \P] {\P};
				\draw[fill] (\P) circle (2pt);
			}

		% \foreach \P in {C,D,E,F} {
		% 		\node[above = 0.2cm of \P] {\P};
		% 		\draw[fill] (\P) circle (2pt);
		% 	}
		\draw[opacity=0] (A) -- (B) node[anchor=west] {$(d)$};
		\draw[opacity=0] (C) -- (D) node[anchor=east] {$(d')$};
		\draw[opacity=0] (E) -- (F) node[anchor=east] {$(d'')$};
	\end{tikzpicture}
\end{exercice}

\end{document}