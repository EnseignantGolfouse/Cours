\documentclass[a4paper,11pt]{article}

\usepackage{../../préambule}

\usetikzlibrary{positioning}

\addtolength{\oddsidemargin}{-0.5cm}
\addtolength{\evensidemargin}{-0.5cm}
\addtolength{\textwidth}{1cm}

\makeatletter
\renewcommand{\maketitle}{%
{\tiny colle dans ton cahier d'exercices}

	\begin{center}
		\LARGE
		\myuline{\@title}
		\vspace{1em}
	\end{center}
}
\makeatother

\title{Construction d'une figure}
\date{}
\author{}

\begin{document}

\maketitle

\begin{attention}[frametitle={⚠}]
	Fais les constructions dans ton cahier d'exercices.
\end{attention}

\begin{exercice}
	\begin{enumerate}
		\item Place un point $O$ au milieu de la feuille.
		\item Construit un cercle de centre $O$, et de rayon 8 carreaux.
		\item Trace deux lignes perpendiculaires passant par $O$, et marque les points où elles intersectent le cercle.
		\item Trace les deux nouveaux axes de symétrie de cette figure, et marque les points où ils intersectent le cercle.
	\end{enumerate}

	\newcommand{\mysize}{8}

	\begin{tikzpicture}
		\coordinate (O) at (0,0);
		\coordinate (A1) at (\mysize,0);
		\coordinate[rotate around={45:(O)}] (B1) at (A1);
		\coordinate (A2) at (0,-\mysize);
		\coordinate[rotate around={45:(O)}] (B2) at (A2);
		\coordinate (A3) at (-\mysize,0);
		\coordinate[rotate around={45:(O)}] (B3) at (A3);
		\coordinate (A4) at (0,\mysize);
		\coordinate[rotate around={45:(O)}] (B4) at (A4);

		\foreach \P in {O} {
				\node[right = 0.2cm of \P] {\P};
			}
		\foreach \P in {O,A1,A2,A3,A4,B1,B2,B3,B4} {
				\draw[fill] (\P) circle (2pt);
			}

		\draw (O) circle (\mysize cm);

		\draw ([xshift=0.5cm] A1) -- ([xshift=-0.5cm] A3);
		\draw ([yshift=-0.5cm] A2) -- ([yshift=0.5cm] A4);
		\draw ([xshift=0.3cm,yshift=0.3cm] B1) -- ([xshift=-0.3cm,yshift=-0.3cm] B3);
		\draw ([xshift=0.3cm,yshift=-0.3cm] B2) -- ([xshift=-0.3cm,yshift=0.3cm] B4);

		\draw (B1) -- (A1) -- (B2) -- (A2) -- (B3) -- (A3) -- (B4) -- (A4) -- cycle;
	\end{tikzpicture}
\end{exercice}

\end{document}