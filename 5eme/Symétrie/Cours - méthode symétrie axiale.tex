\documentclass[a4paper,12pt]{article}

\usepackage{préambule}

\begin{document}
\begin{methode}
	% Takes 'point_x', 'axis_x'
	\newcommand\mySymetricToAxis[2]{
		\directlua{
			local arg1 = token.scan_argument()
			local arg2 = token.scan_argument()
			tex.print(2 * tonumber(arg2) - tonumber(arg1))
		}
		{#1}{#2}
	}
	Pour faire le symétrique par rapport à un axe :
	\begin{enumerate}
		\item Pour chaque point sur la figure d'origine, trace une ligne passant par ce point, \textbf{perpendiculaire} à l'axe de symétrie.

		      \newcommand{\Ax}{0}
		      \newcommand{\Ay}{2}
		      \newcommand{\Bx}{1}
		      \newcommand{\By}{1}
		      \newcommand{\Cx}{0}
		      \newcommand{\Cy}{0}

		      \newcommand{\mycoordinates}{
			      \coordinate (A) at (\Ax,\Ay);
			      \coordinate (B) at (\Bx,\By);
			      \coordinate (C) at (\Cx,\Cy);
			      \coordinate (A_mid) at (3,\Ay);
			      \coordinate (B_mid) at (3,\By);
			      \coordinate (C_mid) at (3,\Cy);
			      \coordinate (A') at (\mySymetricToAxis{\Ax}{3},\Ay);
			      \coordinate (B') at (\mySymetricToAxis{\Bx}{3},\By);
			      \coordinate (C') at (\mySymetricToAxis{\Cx}{3},\Cy);
		      }

		      \begin{tikzpicture}
			      \mycoordinates

			      \draw[fill=gray] (A) circle (2pt) node[anchor=south east] {A};
			      \draw[fill=gray] (B) circle (2pt) node[anchor=south west] {B};
			      \draw[fill=gray] (C) circle (2pt) node[anchor=south east] {C};
			      \draw (A) -- (B);
			      \draw (B) -- (C);
			      \draw[color=red] (3,-1) -- (3,3);

			      \draw[dashed] (-1,\Ay) -- (7,\Ay);

			      \draw ([xshift=-0.2cm] A_mid) -- ++(0,0.2) -- ++(0.2,0);
		      \end{tikzpicture}
		\item Puis, place un point \textbf{à la même distance} de l'autre côté de l'axe.

		      \begin{tikzpicture}
			      \mycoordinates

			      \draw[fill=gray] (A) circle (2pt) node[anchor=south east] {A};
			      \draw[fill=gray] (B) circle (2pt) node[anchor=south west] {B};
			      \draw[fill=gray] (C) circle (2pt) node[anchor=south east] {C};
			      \draw (A) -- (B);
			      \draw (B) -- (C);
			      \draw[color=red] (3,-1) -- (3,3);

			      \draw[dashed] (-1,\Ay) -- (7,\Ay);

			      \draw ([xshift=-0.2cm] A_mid) -- ++(0,0.2) -- ++(0.2,0);

			      \draw (1.4,2.2) -- (1.6,1.8);
			      \draw (4.3,2.2) -- (4.5,1.8);

			      \draw[fill=gray] (A') circle (2pt) node[anchor=south west] {A'};
		      \end{tikzpicture}
		\item Fait de même pour les autres point :

		      \begin{tikzpicture}
			      \mycoordinates

			      \draw[fill=gray] (A) circle (2pt) node[anchor=south east] {A};
			      \draw[fill=gray] (B) circle (2pt) node[anchor=south west] {B};
			      \draw[fill=gray] (C) circle (2pt) node[anchor=south east] {C};
			      \draw (A) -- (B);
			      \draw (B) -- (C);
			      \draw[color=red] (3,-1) -- (3,3);

			      \draw[dashed] (-1,\Ay) -- (7,\Ay);
			      \draw[dashed] (-1,\By) -- (7,\By);
			      \draw[dashed] (-1,\Cy) -- (7,\Cy);

			      \draw ([xshift=-0.2cm] A_mid) -- ++(0,0.2) -- ++(0.2,0);
			      \draw ([xshift=-0.2cm] B_mid) -- ++(0,0.2) -- ++(0.2,0);
			      \draw ([xshift=-0.2cm] C_mid) -- ++(0,0.2) -- ++(0.2,0);

			      \draw (1.4,2.2) -- (1.6,1.8);
			      \draw (4.3,2.2) -- (4.5,1.8);

			      \draw (2,1.2) -- (2.2,0.8);
			      \draw (2.1,1.2) -- (2.3,0.8);
			      \draw (3.7,1.2) -- (3.9,0.8);
			      \draw (3.8,1.2) -- (4,0.8);

			      \draw (1.2,0.2) -- (1.4,-0.2);
			      \draw (1.3,0.2) -- (1.5,-0.2);
			      \draw (1.4,0.2) -- (1.6,-0.2);
			      \draw (4.3,0.2) -- (4.5,-0.2);
			      \draw (4.4,0.2) -- (4.6,-0.2);
			      \draw (4.5,0.2) -- (4.7,-0.2);

			      \draw[fill=gray] (A') circle (2pt) node[anchor=south west] {A'};
			      \draw[fill=gray] (B') circle (2pt) node[anchor=south east] {B'};
			      \draw[fill=gray] (C') circle (2pt) node[anchor=south west] {C'};
		      \end{tikzpicture}
		\item Enfin, relie les points qui étaient reliés sur la figure d'origine.

		      \begin{tikzpicture}
			      \mycoordinates

			      \draw[fill=gray] (A) circle (2pt) node[anchor=south east] {A};
			      \draw[fill=gray] (B) circle (2pt) node[anchor=south west] {B};
			      \draw[fill=gray] (C) circle (2pt) node[anchor=south east] {C};
			      \draw (A) -- (B);
			      \draw (B) -- (C);
			      \draw[color=red] (3,-1) -- (3,3);

			      \draw[dashed] (-1,\Ay) -- (7,\Ay);
			      \draw[dashed] (-1,\By) -- (7,\By);
			      \draw[dashed] (-1,\Cy) -- (7,\Cy);

			      \draw ([xshift=-0.2cm] A_mid) -- ++(0,0.2) -- ++(0.2,0);
			      \draw ([xshift=-0.2cm] B_mid) -- ++(0,0.2) -- ++(0.2,0);
			      \draw ([xshift=-0.2cm] C_mid) -- ++(0,0.2) -- ++(0.2,0);

			      \draw (1.4,2.2) -- (1.6,1.8);
			      \draw (4.3,2.2) -- (4.5,1.8);

			      \draw (2,1.2) -- (2.2,0.8);
			      \draw (2.1,1.2) -- (2.3,0.8);
			      \draw (3.7,1.2) -- (3.9,0.8);
			      \draw (3.8,1.2) -- (4,0.8);

			      \draw (1.2,0.2) -- (1.4,-0.2);
			      \draw (1.3,0.2) -- (1.5,-0.2);
			      \draw (1.4,0.2) -- (1.6,-0.2);
			      \draw (4.3,0.2) -- (4.5,-0.2);
			      \draw (4.4,0.2) -- (4.6,-0.2);
			      \draw (4.5,0.2) -- (4.7,-0.2);

			      \draw[fill=gray] (A') circle (2pt) node[anchor=west] {A'};
			      \draw[fill=gray] (B') circle (2pt) node[anchor=south east] {B'};
			      \draw[fill=gray] (C') circle (2pt) node[anchor=west] {C'};
			      \draw (A') -- (B');
			      \draw (B') -- (C');
		      \end{tikzpicture}
	\end{enumerate}
\end{methode}
\end{document}