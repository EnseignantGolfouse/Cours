\documentclass[a4paper,12pt]{article}

\usepackage{../préambule}
\usepackage{clipboard}

\begin{document}

\Copy{Methode}{
	\begin{methode}
		Pour faire le symétrique d'un point A par rapport à un centre O : \vspace{1em}

		\begin{tikzpicture}
			\coordinate (O) at (0,0);


			\draw[fill] (O) circle (2pt) node[anchor=south west] {O};
			\draw[fill] (O) ++(-1,1) circle (2pt) node[anchor=north east] {A};
		\end{tikzpicture}
		\begin{enumerate}
			\item Trace la droite, qui part du point et passe par le centre :

			      \begin{tikzpicture}
				      \coordinate (O) at (0,0);
				      \coordinate (A) at ([xshift=-1cm,yshift=1cm] O);


				      \draw[fill] (O) circle (2pt) node[anchor=south west] {O};
				      \draw[fill] (A) circle (2pt) node[anchor=north east] {A};
				      \draw[dashed] (A) ++(-0.5,0.5) -- (O) -- ++(1.5,-1.5);
			      \end{tikzpicture}
			\item Place le symétrique du point A \textbf{à égale distance de O} :

			      \begin{tikzpicture}
				      \coordinate (O) at (0,0);
				      \coordinate (A) at (-1,1);
				      \coordinate (A_sym) at (1,-1);


				      \draw[fill] (O) circle (2pt) node[anchor=south west] {O};
				      \draw[fill] (A) circle (2pt) node[anchor=north east] {A};
				      \draw[fill] (A_sym) circle (2pt) node[anchor=south west] {A'};
				      \draw[dashed] (A) ++(-0.5,0.5) -- (O) -- ++(1.5,-1.5);

				      \draw (-0.6,0.4) -- ++(0.2,0.2);
				      \draw (-0.55,0.35) -- ++(0.2,0.2);

				      \draw[rotate around={180:(O)}] (-0.6,0.4) -- ++(0.2,0.2);
				      \draw[rotate around={180:(O)}] (-0.55,0.35) -- ++(0.2,0.2);
			      \end{tikzpicture}

			      Tu peux utiliser un \textbf{compas} pour cette étape !
		\end{enumerate}
	\end{methode}
}

\vspace{1.5em}
\hrule
\vspace{1em}

\Paste{Methode}

\end{document}