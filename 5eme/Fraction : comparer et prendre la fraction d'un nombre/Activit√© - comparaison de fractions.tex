\documentclass[a4paper,12pt,landscape,twocolumn]{article}

\usepackage{préambule}
\usepackage{clipboard}

\geometry{left=1cm}

\setlength{\columnsep}{0.8cm}

\newcommand{\DessineCorrection}{0}
\ifthenelse{\DessineCorrection=1}{
	\renewcommand{\phantom}[1]{{\color{red}#1}}
}

\TitreDActivite{Activité : Comparaison de fractions}

\begin{document}

\Copy{Activité}{
	\maketitle

	\begin{exercice}
		Deux amies se partagent un paquet de cartes.

		Élise en prend $\dfrac{7}{12}$. Fanny en prend $\dfrac{5}{12}$.

		Qui a pris le plus de cartes ?

		\vspace{1em}\dotfill \phantom{Élise}
	\end{exercice}

	\begin{exercice}
		Trois amis se partagent un nombre de billes, que l'on ne connait pas.

		Gaétan prend $\dfrac{1}{4}$ des billes. \vspace{0.5em}

		Hayan prend $\dfrac{3}{10}$ des billes. \vspace{0.5em}

		Isaac prend $\dfrac{9}{40}$ des billes. \vspace{0.5em}

		Qui a eu le plus de billes ? Qui en a eu le moins ? Comment as-tu trouvé ?

		\vspace{1em}\dotfill \phantom{}

		\vspace{1em}\dotfill

		\phantom{Hayan, puis Isaac, car 3/10 > 1/4 > 9/40}

		\awesomebox[violet]{2pt}{\faRocket}{violet}{
			\textbf{Pour aller plus loin}

			Quelle part de billes reste-t-il ?
		}
	\end{exercice}
}

\newpage \setcounter{exercice}{0}

\Paste{Activité}

\end{document}