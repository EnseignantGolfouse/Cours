\documentclass[a4paper,11pt]{article}

\usepackage{préambule}

\makeatletter
\renewcommand{\maketitle}{%
{\scriptsize colle dans ton cahier d'exercices}
	\begin{center}
		\LARGE
		\myuline{\@title}
		\vspace{0.5em}
	\end{center}
}
\makeatother

\title{Activité : Destruction de fraction}
\date{}
\author{}

\begin{document}

\maketitle

\begin{enonce}
	On va faire un jeu qui se joue à deux :

	On dispose d'une fraction, ainsi que de plusieurs nombres. Par exemple, $\cfrac{5}{6}$ et $1, 2, 3$.

	Chacun son tour, un joueur peut:
	\begin{enumerate}
		\item Simplifier la fraction si il le souhaite, puis
		\item Utiliser un des nombres pour l'enlever au numérateur ou au dénominateur.
	\end{enumerate}

	Voici un exemple de partie avec la fraction $\cfrac{5}{6}$ et les nombres $1, 2, 3$ :
	\begin{itemize}
		\item (Joueur 1) Enlève $1$ à $5$ : la fraction devient $\cfrac{4}{6}$.
		\item (Joueur 2) Simplifie la fraction par $2$ : $\cfrac{4}{6} = \cfrac{2}{3}$
		
		Puis, enlève $2$ à $2$ : la fraction devient $\cfrac{0}{3}$, c'est gagné !
	\end{itemize}
\end{enonce}


\end{document}