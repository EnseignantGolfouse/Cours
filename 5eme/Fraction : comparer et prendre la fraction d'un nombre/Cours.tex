\documentclass[a4paper,11pt]{article}

\usepackage{préambule}

\title{Chapitre 7 : Fractions}
\date{}
\author{}

\begin{document}

\maketitle

\section{L'écriture fractionnaire}

\begin{cours}[écriture fractionnaire]
	Soient a et b deux nombres, avec b non égal à 0. Le quotient de a par b est le nombre qui, multiplié par b, donne a.

	On peut le noter :
	\begin{itemize}
		\item a $÷$ b : c'est l'écriture \textbf{décimale}.
		\item $\cfrac{\text{a}}{\text{b}}$ : c'est l'écriture \textbf{fractionnaire}.

		      a est le \textbf{numérateur}.

		      b est le \textbf{dénominateur}.
	\end{itemize}
\end{cours}

\importantbox{
	On ne peut \textbf{jamais} diviser par 0.
}

\begin{exemple}
	Le quotient de 8 par 9 est $\cfrac{8}{9}$, et on a $\cfrac{8}{9} × 9 = 8$.
\end{exemple}

\begin{cours}[Fractions]
	Lorsque a et b sont des nombres \textit{entiers}, on dit que $\cfrac{\text{a}}{\text{b}}$ est une \textbf{fraction}.
\end{cours}

\section{Simplifier des fractions}

\begin{cours}
	Si on \textbf{multiplie} ou \textbf{divise} le numérateur \textbf{et} le dénominateur d'un quotient par le \textbf{même} nombre (différent de 0), la valeur du quotient reste la même.

	Si a, b, et k sont trois nombres, avec b ≠ 0 et k ≠ 0, alors

	$$ \frac{\text{a}}{\text{b}} = \frac{\text{a × k}}{\text{b × k}} = \frac{\text{a }÷\text{ k}}{\text{b }÷\text{ k}} $$
\end{cours}

\begin{exemple}
	\begin{align*}
		\frac{24}{30} & = \frac{24 ÷ 6}{30 ÷ 6} = \frac{4}{5}   &
		\frac{3,5}{6} & = \frac{3,5 × 2}{6 × 2} = \frac{7}{12}
	\end{align*}
\end{exemple}

\begin{cours}
	Pour \textbf{simplifier} une fraction, il faut écrire une autre fraction qui lui est égale, mais dont le numérateur et le dénominateur sont plus petits.
\end{cours}

\begin{exemple}
	Pour simplifier $\cfrac{36}{15}$ :
	\begin{itemize}
		\item 36 et 15 sont divisible par 3.
		\item Donc on a $\cfrac{36}{15} = \cfrac{36 ÷ 3}{15 ÷ 3} = \cfrac{12}{5}$
	\end{itemize}
\end{exemple}

\notebox{
	Pour simplifier au maximum une fraction, on peut utiliser le \textbf{PGCD} (vu au chapitre 1 : multiples, diviseurs, nombres premiers).

	Par exemple, pour simplifier $\cfrac{18}{42}$, on calcule :
	\begin{itemize}
		\item 18 = 2 × 3 × 3, et 42 = 2 × 3 × 7, donc PGCD(18, 42) = 2 × 3 = 6.
		\item Donc on a $\cfrac{18}{42} = \cfrac{18 ÷ 6}{42 ÷ 6} = \cfrac{3}{7}$.
	\end{itemize}
}

\end{document}