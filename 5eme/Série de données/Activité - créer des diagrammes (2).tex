\documentclass[a4paper,12pt,landscape,twocolumn]{article}

\usepackage{préambule}
\usepackage{clipboard}

\geometry{left=1cm}
\setlength{\columnsep}{0.8cm}

\TitreDActivite{Créer des diagrammes}

\renewcommand{\arraystretch}{1.2}

\begin{document}

\Copy{Circulaire}{
	\maketitle

	On a interrogé 100 personnes pour savoir quel sport ils faisaient, et voici leurs réponses :

	\begin{center}
		\begin{tabular}{|l|c|c|c|c|c|}
			\hline
			Sport    & Natation & Football & Course & Escalade & Tennis
			\\ \hline
			Effectif & 35       & 20       & 15     & 25       & 5
			\\ \hline
		\end{tabular}
	\end{center}

	Indique la part de chaque sport sur le diagramme circulaire suivant :
	\vspace{2em}

	\begin{center}
		\begin{tikzpicture}
			\draw (0,0) circle (4);
			\draw (0,0) -- (0,4) arc (90:108:4) -- (0,0);
			\draw (-0.5,3.5) -- ++(-1,1) -- ++(-2,0) node[left] {Tennis};
		\end{tikzpicture}
	\end{center}

	\vspace{1em}

	\uline{Indice} : le cercle totalise 360°. Quel angle doit alors faire chaque part ? Tu peux utiliser un tableau de proportionnalité.
}

\newpage
\Paste{Circulaire}

\end{document}