% !TEX root = ./Cours.tex
\documentclass[../€Cours-complet/Cours-complet]{subfiles}

\titleorchapter{Calcul littéral}{9}

\begin{document}

\maketitleCours

\section{Expression littérale}

\begin{cours}[Expression littérale]
	Une \textbf{expression littérale} est une expression dans laquelle un ou plusieurs nombres sont désignés par des lettres.
\end{cours}

\begin{exemple}
	$$ A = 2 × x + 3 \hspace{8em} B = x + 2 × y $$

	sont des expressions littérales.
\end{exemple}

\begin{cours}[Simplification d'écriture]
	Pour alléger l'écriture d'une expression littérale, le signe × peut être supprimé dans certains cas :
	\begin{itemize}
		\item Entre un nombre et une lettre :

		      $3 × a = 3a$

		      $20 − 5 × x = 20 − 5 x$
		\item Entre 2 lettres :
		\item
		      $x × y = xy$

		      $2 × a × b = 2 ab$
		\item Entre un nombre et une parenthèse :

		      $2 × (x + 3) = 2(x + 3)$

		      $5 × (3 × x - 1) = 5(3x − 1)$
		\item Entre une lettre et une parenthèse :

		      $x × (7+ y) = x (7+ y)$

		      $6 × x × (2 + y)= 6x(2 + y)$
		\item Entre 2 parenthèses :

		      $(5 + x) × (3−2 y) = (5 + x) (3−2 y)$
		\item $1x$ s'écrit simplement $x$
	\end{itemize}
\end{cours}

\begin{cours}[Développer, factoriser]
	\begin{itemize}
		\item \textbf{Développer} une expression littérale, c’est l’écrire comme une somme de termes.
		\item \textbf{Factoriser} une expression littérale, c’est l’écrire comme un produit de facteurs.
	\end{itemize}
\end{cours}

\begin{exemple}
	\begin{itemize}
		\item $A = 7x + 3(x+2)$ est une forme quelconque.
		\item $B = 3z + 5 - 2z + 9$ est une forme développée.
		\item $C = 5(g - 7)$ est une forme factorisée.
		\item $D = (a - 3) (a + 4)$ est une forme factorisée.
	\end{itemize}
\end{exemple}

\begin{cours}[Utilisation d'une expression littérale]
	Pour utiliser une expression littérale avec certaines valeurs, on \textit{remplace} dans l'expression toutes les \textit{lettres} par leurs \textit{valeurs}.
\end{cours}

\begin{exemple}
	L'aire $𝒜$ d'un rectangle peut s'écrire comme le produit de sa longueur $L$ et de sa largeur $𝑙$ :
	$$ 𝒜 = L × 𝑙 $$

	Si on veut calculer l'aire d'un rectangle de longueur $6$ et de largeur $4$, on doit donc remplacer $L$ par $6$ et $𝑙$ par $4$ :
	\begin{align*}
		𝒜 & = L × 𝑙 \\
		𝒜 & = 6 × 4 \\
		𝒜 & = 24    \\
	\end{align*}
\end{exemple}

\section{Tester une égalité}

\begin{cours}[Égalité]
	\begin{itemize}
		\item Une \textbf{égalité} est constituée de deux \textbf{membres} séparés par un signe =.
		\item Une égalité est \textbf{vraie} si les deux membres ont la même valeur.
	\end{itemize}
\end{cours}

\begin{exemple}
	$$ \underbrace{3 × 6} \hspace{4.5em} = \hspace{4.5em} \underbrace{13 + 5} $$
	\begin{center}
		membre de gauche \hspace{3em} membre de droite
	\end{center}

	Cette égalité est \textit{vraie} car les deux membres valent $18$. \vspace{1em}

	$$ \underbrace{1 + 2 + 18} \hspace{2.5em} = \hspace{5.5em} \underbrace{5 × 4} $$
	\begin{center}
		membre de gauche \hspace{3em} membre de droite
	\end{center}

	Cette égalité est \textit{fausse} car le membre de gauche vaut $21$, tandis que le membre de droite vaut $20$.
\end{exemple}

\begin{cours}[Égalité avec des lettres]
	Si une égalité contient des lettres, elle peut être \textit{vraie} pour certaines valeurs, et \textit{fausse} pour d'autres.
\end{cours}

\begin{exemple}
	Considérons l'égalité $x + 6 = 19$.
	\begin{itemize}
		\item Si $x = 8$, cette égalité est fausse : on a $14$ à gauche et $19$ à droite.
		\item Si $x = 13$, cette égalité est vraie : on a $19$ des deux côtés.
	\end{itemize}
\end{exemple}

\section*{Résumé / carte mentale}

\myimportantbox[frametitle={Expression littérale}]{cyan}{
	Les lettres désignent des nombres.

	\textit{Exemple} : $x + 2$
}

\myimportantbox[frametitle={Calcul de la valeur d'une expression littérale}]{orange}{
	On remplace les lettres par leurs valeurs.

	\textit{Exemple} : Avec $x = 1$,

	\begin{align*}
		x + 2 & = 1 + 2 \\
		      & = 3
	\end{align*}
}

\myimportantbox[frametitle={Égalité}]{green}{
	$$ \underset{\text{membre de gauche}}{G} = \underset{\text{membre de droite}}{D} $$
}

\myimportantbox[frametitle={Tester une égalité}]{lime}{
	Calcul de $G$ puis calcul de $D$.

	\begin{itemize}
		\item Si Résultat $G$ = Résultat $D$ alors l'égalité est \textbf{vraie}.
		\item Si Résultat $G$ ≠ Résultat $D$ alors l'égalité est \textbf{fausse}.
	\end{itemize}
}

\end{document}