% !TEX root = ./Cours.tex
\documentclass[../€Cours-complet/Cours-complet]{subfiles}

\usetikzlibrary{calc,positioning}

\titleorchapter{Solides de l'espace, volume}{12}

\begin{document}

\maketitleCours

\begin{cours}[parallélépipède rectangle]
	Un \textbf{parallélépipède rectangle} (ou \textbf{pavé droit}) est un solide (en trois dimensions) qui a
	\begin{itemize}
		\item 6 faces rectangulaires.
		\item 12 arètes.
		\item 8 sommets.
	\end{itemize}


	\newcommand{\rectWidth}{3}
	\newcommand{\rectHeight}{2}
	\begin{center}
		\begin{tikzpicture}
			\coordinate (A) at (0,0);
			\coordinate (B) at ($(A) + (\rectWidth,0)$);
			\coordinate (C) at ($(A) + (\rectWidth,\rectHeight)$);
			\coordinate (D) at ($(A) + (0,\rectHeight)$);

			\coordinate (E) at ($(A) + (1,1)$);
			\coordinate (F) at ($(E) + (\rectWidth,0)$);
			\coordinate (G) at ($(E) + (\rectWidth,\rectHeight)$);
			\coordinate (H) at ($(E) + (0,\rectHeight)$);

			\draw[fill=blue!30,draw=transparent] (B) -- (F) -- (G) -- (C) -- (B);

			\draw (B) -- (C) -- (D) -- (A) -- (B) -- (F) -- (G) -- (H) -- (D)
			(C) -- (G);
			\draw[dotted] (E) -- (A)
			(E) -- (H)
			(E) -- (F);

			\foreach \p / \pos in {
					A/below left,
					B/below right,
					C/above left,
					D/above left,
					E/below,
					F/below right,
					G/above right,
					H/above} {
					\node[\pos] at (\p) {\p};
				}

			\draw[draw=green,very thick] (D) -- (H);
			\node at (A) {\color{red}∙};

			\draw[->] ($(A) - (2,0)$) node[left] {{\color{red}Sommet} A} -- ($(A) - (0.5,0)$);
			\draw[->] ($(D) + (-1,1.5)$) node[left] {{\color{green}Arête} [DH]} -- ($(D) + (0.3,0.7)$);
			\draw[->] ($(F) + (1,0.5)$) node[right] {{\color{blue}Face} BFGC} -- ($(F) + (-0.5,0.5)$);
		\end{tikzpicture}
	\end{center}
\end{cours}

\begin{cours}[volume d'un parallélépipède rectangle]
	Le \textbf{volume $𝒱$} d'un parallélépipède rectangle de \uline{longueur} \textit{L}, de \uline{longueur} $𝑙$ et de \uline{hauteur} \textit{h} est :
	$$ 𝒱 = \textit{L} × l × \textit{h} $$
\end{cours}

\end{document}