\documentclass[a4paper,11pt]{article}

\usepackage{préambule}

\newmdenv[style=greenstyle]{notation}

\begin{document}

\begin{center}
	\LARGE

	\uline{Multiples de 3}
\end{center}

Tu sais peut-être déjà que, pour qu'un nombre soit multiple de 3, il faut (et suffit) que la \textbf{somme des chiffres} de ce nombre soit aussi multiple de 3. Nous allons ici en faire la preuve !

\begin{greybox}[frametitle={Remarque :}]
	On ne peut pas utiliser le critère de divisibilité par 3 tout au long de cette fiche, car \textbf{c'est ce qu'on essaie de montrer !}
\end{greybox}

\section{Puissance de 10}

\begin{itemize}
	\item Calcule $10 - 1 = ....$

	      Est-ce un multiple de 3 ?
	\item Fait de même pour $100 - 1 = ......$ et $1000 - 1 = ........$ .

	      \uline{Peux-tu faire une hypothèse ?}
	\item Essaie de prouver cette hypothèse (sans utiliser le critère de divisibilité par 3 !).

	      \renewcommand{\arraystretch}{1.2}
	      \begin{tabular}{|l|}
		      \hline Indice                                                              \\ \hline
		      On sait que $10\ 000 = 10 × 1000$.                                         \\

		      Donc, $10\ 000 - 1 = (1000 - 1) × .... + ....$.                            \\

		      Enfin, tu sais que $(1000 - 1)$ est multiple de 3 : donc, que peux-tu dire \\ de $10\ 000 - 1$ ? Fait de même pour $100\ 000 - 1$.
		      \\ \hline
	      \end{tabular}
\end{itemize}

\begin{notation}[frametitle={Notation}]
	On note $10ᵏ$ pour dire \textbf{un 10 suivi de k zéros}.
\end{notation}

On sait maintenant que quel que soit $k$, $10ᵏ - 1$ est un multiple de 3.

\section{Nombres généraux}

Prenons le nombre $2\ 000\ 000$.

\begin{itemize}
	\item D'après la section précédente, on peut dire que $1\ 000\ 000 - 1 = 3 × n$ (où $n$ est un nombre que nous ne connaissons pas. En tout cas pas moi :)). Donc
	      $$1\ 000\ 000 = .... × n + ....$$
	\item Sachant cela, écrit
	      $$ 2\ 000\ 000 = 3 × .... + .... $$
	\item De même, écrit
	      $$ 3\ 000\ 000 = 3 × .... + .... $$
	      Que remarques-tu ?
\end{itemize}

Prenons maintenant le nombre $5\ 000 \ 010$.
\begin{itemize}
	\item On écrit $1\ 000\ 000 - 1 = 3 × n$ et $10 - 1 = 3 × m$. Donc
	      $$ 1\ 000\ 000 = .... × n + .... \text{ \hspace{2em}et\hspace{2em} } 10 = .... × m + .... $$
	\item Ainsi, on peut écrire
	      $$ 5\ 000 \ 010 = 3 × .... + .... $$
\end{itemize}
Conclus.

\end{document}