\documentclass[a4paper,11pt]{article}

\usepackage{préambule}

\addtolength{\oddsidemargin}{-1cm}
\addtolength{\evensidemargin}{-1cm}
\addtolength{\textwidth}{2cm}
\addtolength{\topmargin}{-1cm}
\addtolength{\textheight}{1cm}

\title{Scratch : Évite les méchants !}
\date{}
\author{}

\begin{document}

\maketitle

\begin{attention}[frametitle={\huge ⚠}]
	Lit bien la fiche pour comprendre les étapes !
\end{attention}

\begin{center}
	\large
	On va faire un jeu où il faudra éviter des méchants qui bougent autour de nous.
\end{center}

\section{Des lutins}

\begin{vocabulaire}[Lutins]
	Les objets de scratch (qui peuvent bouger, parler, changer de costumes, etc...) s'appellent des \textbf{lutins}. Ils peuvent aussi être appelés objets, ou encore scripts.
\end{vocabulaire}

Crée :
\begin{itemize}
	\item Un lutin pour le \myuline{personnage} qu'on va contrôler.
	\item Un lutin pour le \myuline{méchant}.
\end{itemize}

Donne-leur des noms pour les reconnaître.

\section{Il est vivant !}

En te basant sur la dernière séance, crée un script pour pouvoir bouger le \myuline{personnage} avec la souris OU le clavier.

\begin{attention}
	Si tu a du mal pour cette partie, appelle le professeur.
\end{attention}

Choisit maintenant un costume pour le \myuline{méchant}.

Puis, fait le avancer en ligne droite, et rebondir sur les bords de l'écran. \textbf{Lit bien les blocks 'Mouvement'} : il y a tout ce dont tu as besoin !

\begin{attention}
	Vérifie que tu peut bouger ton \myuline{personnage} \textbf{en même temps} que le \myuline{méchant} se déplace.
\end{attention}

\section{Pas touche}

Le but du jeu est d'éviter le \myuline{méchant} ! On doit donc perdre lorsqu'il nous touche.

Pour cela, on va utiliser le block \squared{Touche \_ ?} dans 'Capteurs'.

\begin{attention}
	Si tu as du mal, rappelles-toi : on veut que \textbf{si} le \myuline{méchant} nous touche, \textbf{alors} on a perdu. Essaye de trouver un block utilisant ces deux mots...
\end{attention}

\section*{Bonus}

\begin{itemize}
	\item Compte le temps qui passe, pour essayer de durer le plus longtemps possible !
	\item Sans créer d'autres lutins, crée plusieurs méchants.
	\item Ajoute de plus en plus de méchants quand le temps passe 😈
\end{itemize}

\end{document}