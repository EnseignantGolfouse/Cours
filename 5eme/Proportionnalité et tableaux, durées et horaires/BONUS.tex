\documentclass[a4paper,11pt]{article}

\usepackage{../préambule}
\usetikzlibrary{positioning}

\newmdenv[style=redstyle]{attention}

\title{Un problème explosif}
\date{}
\author{}

\begin{document}

\begin{center}
	\LARGE

	Un problème explosif
\end{center}

\squared{\footnotesize
Ce problème est tiré du TFJM², une compétition mathématique pour lycéen⋅nes. Il a été adapté.
}

\vspace{1em}

L'agent secret 1234 doit désamorcer une bombe. Mais il y a un souci : la bombe est protégé par un cadenas à chiffres !

\begin{itemize}
	\item Le cadenas possède 4 roues, et une seule combinaison peut l'ouvrir.
	\item Chaque roue a des chiffres de 0 à 9.
	\item On ne peut tourner une roue que dans un sens : 0 → 1 → 2 → ⋯ → 9 → 0 → ⋯.
	\item ATTENTION ! Si on rentre une combinaison déjà essayée, la bombe explose automatiquement !
\end{itemize}

Lorsque 1234 arrive, le cadenas est sur la position 0-0-0-0.

Son but est donc de tester le plus de combinaisons possibles, en espérant tomber sur la bonne.

\begin{enumerate}
	\item On suppose dans cette question que les trois derniers chiffres de la combinaison sont 0.

	      Comment 1234 doit-il procéder pour désamorcer la bombe ?
	\item Comment doit-il procéder si il sait que les deux derniers chiffres de la combinaison sont 0 ?
	\item Peut-il désamorcer la bombe à coup sûr si il ne connait aucun chiffre ?
\end{enumerate}


\end{document}