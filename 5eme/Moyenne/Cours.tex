% !TEX root = ./Cours.tex
\documentclass[../€Cours-complet/Cours-complet]{subfiles}

\titleorchapter{Série de données}{10}
% Moyenne (moyenne pondérée) et représentations

\renewcommand{\arraystretch}{1.3}

\begin{document}

\maketitleCours

\begin{cours}[Effectif]
	Dans une série de données :
	\begin{itemize}
		\item L'\textbf{effectif} d'une donnée est le nombre de fois où cette donnée apparait.
		\item L'\textbf{effectif total} est la somme de tous les effectifs.
	\end{itemize}
\end{cours}

\begin{exemple}
	Voici les couleurs de cheveux des élèves dans une classe :

	\begin{center}
		\begin{tabular}{|c|c|c|c|}
			\hline
			Taille          & Blond & Brun & Noir
			\\ \hline
			Nombre d'élèves & $5$   & $12$ & $7$
			\\ \hline
		\end{tabular}
	\end{center}

	L'\textit{effectif} des élèves ayant les cheveux blonds est $5$.

	L'\textit{effectif total} est $5 + 12 + 7 = 24$.
\end{exemple}

\begin{cours}[Moyenne]
	Si les données sont des nombres, la \textbf{moyenne} de la série de données est égale à la somme de toutes ces données, divisées par l'effectif total.
\end{cours}

\begin{exemple}
	Si les cinq notes du semestre d'un élève sont $11$, $12$, $10$, $16$ et $18$, sa moyenne est

	$$ \dfrac{11 + 12 + 10 + 15 + 17}{5} = \dfrac{65}{5} = 13 $$
\end{exemple}

\begin{cours}[Fréquence]
	La \textbf{fréquence} d'une donnée est obtenue en divisant son effectif par l'effectif total :
	\begin{center}
		fréquence d'une donnée = $\dfrac{\text{effectif de la donnée}}{\text{effectif total}}$
	\end{center}
\end{cours}

\end{document}