\documentclass[a4paper,12pt]{article}

\usepackage{préambule}

\mdfdefinestyle{hypothesestyle}{
    style=redstyle,
    frametitle={Hypothèse},
}
\newmdenv[style=hypothesestyle]{hypothese}

\newcommand{\DessineCorrection}{0}
\ifthenelse{\DessineCorrection=1}{
	\newcommand{\correction}[1]{{\color{red}#1}}
}{
	\newcommand{\correction}[1]{\dotfill}
}

\renewcommand{\baselinestretch}{1.2}

\TitreDActivite{Activité : contruction de triangles}

\begin{document}

\maketitle

Dans l'activité précédente, on a formulé l'hypothèse :

\begin{hypothese}
	\begin{enumerate}
		\item Si on a trois longueurs, telles que \correction{la somme de deux des longueurs}

		      \correction{dépasse toujours la troisième},

		      alors on peut construire un triangle dont les mesures des côtés sont ces trois longueurs.
		\item À l'inverse, si \correction{la somme de deux des longueurs est plus petite que la}

		      \correction{troisième},

		      alors on ne peut pas construire le triangle.
	\end{enumerate}
\end{hypothese}

\squared{Dans ton cahier d'exercices :}

\subsection*{Partie 1}

Choisit trois longueurs vérifiant le point 1. de l'hypothèse.

\begin{itemize}
	\item Place deux points A et B, telle que leur écart est la \myuline{troisième} longueur.
	\item Si on souhaite placer le point C, telle que la distance [AC] soit la \myuline{première} longueur, dessine \textbf{tous} les endroits où pourrait se trouver C. (Indice : utilise un compas).
	\item Fait de même pour que la distance [BC] soit la \myuline{deuxième} longueur.
	\item Que remarque-t'on ?
\end{itemize}

\subsection*{Partie 2}

Choisit trois longueurs vérifiant le point 2. de l'hypothèse.

\begin{itemize}
	\item Place deux points A et B, telle que leur écart est la \myuline{troisième} longueur.
	\item Si on souhaite placer le point C, telle que la distance [AC] soit la \myuline{première} longueur, dessine \textbf{tous} les endroits où pourrait se trouver C. (Indice : utilise un compas).
	\item Fait de même pour que la distance [BC] soit la \myuline{deuxième} longueur.
	\item Que remarque-t'on ?
\end{itemize}

\end{document}