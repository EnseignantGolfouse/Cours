% !TEX root = ./Cours.tex
\documentclass[../€Cours-complet/Cours-complet]{subfiles}

\titleorchapter{Triangles et cercles}{8}

\begin{document}

\maketitleorchapter

\section{Inégalité triangulaire}

\begin{cours}[Inégalité triangulaire]
	Dans un triangle, la longueur d'un côté est \textbf{toujours inférieure} à la somme des longueurs des deux autres côtés.
\end{cours}

\begin{exemple}
	\begin{tikzpicture}
		\coordinate (A) at (0,0);
		\coordinate (B) at (4,3);
		\coordinate (C) at (8,0);
		\coordinate (AB) at (10,2.5);
		\coordinate (BC) at (10,1.5);
		\coordinate (AC) at (10,0.5);

		\draw (A) node[left] {A} -- (B) node[above] {B} -- (C) node[right] {C} -- cycle;
		\node[right] at (AB) {AB = 5cm};
		\node[right] at (BC) {BC = 5cm};
		\node[right] at (AC) {AC = 8cm};
	\end{tikzpicture}

	On a
	\begin{itemize}
		\item AB < BC + AC
		\item BC < AB + AC
		\item AC < AB + BC
	\end{itemize}
\end{exemple}

\section{Médiatrice}

\begin{cours}[Médiatrice]
	La \textbf{médiatrice} d'un segment [AB] est la droite qui
	\begin{itemize}
		\item Passe par le milieu de [AB].
		\item Est perpendiculaire à [AB].
	\end{itemize}

	Tous les points de la médiatrice sont alors \textbf{à égale distance} de A et B.
\end{cours}

\begin{cours}
	Dans un triangle ABC, les médiatrices des trois côtés se rencontrent \textbf{en un seul point}.

	Ce point est alors \textbf{à égale distance} de A, B et C.
\end{cours}

\section{Hauteurs}

\begin{cours}
	Dans un triangle, la \textbf{hauteur issue d'un sommet} est la droite
	\begin{itemize}
		\item passant par ce sommet ;
		\item perpendiculaire au côté opposé.
	\end{itemize}
\end{cours}

\begin{exemple}
	\begin{minipage}{0.58\linewidth}
		\begin{tikzpicture}
			\coordinate (A) at (0,0);
			\coordinate (B) at (8,0);
			\coordinate (C) at (5,3);
			\coordinate (H) at (5,0);

			\foreach \P/\dir in {A/left, B/right, C/above right, H/below right} {
					\node[\dir] at (\P) {\P};
				}

			\draw (A) -- (B) -- (C) -- cycle;
			\draw ($(C) + (0,1)$) -- ($(H) - (0,1)$);
			\draw (H) ++(0.3,0) -- ++(0,0.3) -- ++(-0.3,0);
		\end{tikzpicture}
	\end{minipage}
	\hspace{1em}
	\begin{minipage}{0.35\linewidth}
		Ici, la hauteur issue de C est la droite (CH).
	\end{minipage}
\end{exemple}


\end{document}