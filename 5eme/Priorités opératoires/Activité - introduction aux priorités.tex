\documentclass[a4paper]{article}

\usepackage{../préambule}

\title{Activité : Introduction aux priorités}
\author{}
\date{}

\begin{document}

\maketitle

\begin{exercice*}
	Faire les calculs suivants, et dire de quelle opération il s'agit : \\[0.5em]

	\begin{tabular}{lcl}
		$A = 1 + 2 × 5$       & \phantom{space} & $A$ est ............... \\[3.5em]
		$B = 3,9 - 1,2 - 1$   & \phantom{space} & $B$ est ............... \\[3.5em]
		$C = 2 × 3 + 1,2 × 5$ & \phantom{space} & $C$ est ............... \\[3.5em]
		$D = 8 ÷ 2 - 1,5 × 2$ & \phantom{space} & $D$ est ............... \\[3.5em]
	\end{tabular}
\end{exercice*}

\begin{greybox}[frametitle={RAPPEL : Opérations sur les nombres à virgule}]
	Pour faire une addition/soustraction sur les nombres à virgules :

	Il faut aligner la virgule des nombres qu'on veut additionner ou soustraire, puis faire l'opération normalement.

	Par exemple:
	\renewcommand{\arraystretch}{1.5}
	$$\begin{array}{lllll}
			  & 1 & 2  , & 5 & 0 \\
			+ &   & 3  , & 3 & 6 \\ \cline{0-4}
			  & 1 & 5  , & 8 & 6
		\end{array}$$
	\renewcommand{\arraystretch}{1}

	Pour les multiplications, c'est plus compliqué : il faut compter combien de chiffres après la virgule il y a dans chacun des facteurs. Cela nous indique combien de chiffres il doit y avoir après la virgule dans le résultat.

	Par exemple :
	\renewcommand{\arraystretch}{1.5}
	$$\begin{array}{lllll}
			  &   &      & 2  , & 6 \\
			× &   &      & 7  , & 1 \\ \cline{0-4}
			  & 1 & 8  , & 4    & 6
		\end{array}$$
	\renewcommand{\arraystretch}{1}
\end{greybox}

\end{document}