\documentclass[a4paper,11pt]{article}

\usepackage{../../préambule}

\title{Contrôle n°2 : Priorités opératoires}
\date{15 octobre 2021}
\author{Prénom : ..................}

\makeatletter
\renewcommand{\maketitle}{%
    \topskip1em
	\@author \hfill \@date \\

	\begin{center}
		\begin{huge}
			\@title \\[1em]
		\end{huge}
	\end{center}
}
\makeatother

\begin{document}

\maketitle

\begin{question}[(4 points)] Effectue les calculs suivants en respectant les priorités opératoires :

	\noindent
	\begin{tabular}{p{0.47\textwidth}|p{0.47\textwidth}}
		A = 14 + 39 − 42 + 7 & B = 16 − 4 × 3 + 2 \vspace{8em}  \\
		C = 15 ÷ 5 × 8 ÷ 3   & D = 40 − 3 × 2 × 6  \vspace{8em}
	\end{tabular}
\end{question}

\begin{question}[(4 points)] Effectue les calculs suivants en respectant les priorités opératoires :

	\noindent
	\begin{tabular}{p{0.47\textwidth}|p{0.47\textwidth}}
		E = 30 − (12 + 6)    & F = (3 + 2) × (5 + 4) \vspace{9em}      \\
		G = 3 + (14 − 5) × 3 & H = 37 − [3 × (5 + 2) − 4] \vspace{9em}
	\end{tabular}
\end{question}

\begin{question}[(3 points)] Complète en utilisant $+,-,×,÷$ pour que les égalités soient vraies :

	\vspace{1em}
	\noindent
	\begin{tabular}{p{0.47\textwidth}p{0.47\textwidth}}
		% 14 - 5 + 6
		14 ... 5 ... 6 = 15       &
		% 5 + 5 - 5 ÷ 5
		5 ... 5 ... 5 ... 5 = 9 \vspace{0.5em} \\
		% 4 + 7 × 10
		4 ... 7 ... 10 = 74 &
	\end{tabular}
	\vspace{1em}

	Complète en utilisant $+,-,×,÷$ \textbf{et les parenthèses} pour que les égalités soient vraies :

	\vspace{1em}
	\noindent
	\begin{tabular}{p{0.47\textwidth}p{0.47\textwidth}}
		% (3 + 2) × 5
		3 ... 2 ... 5 = 25       &
		% 4 × (6 - 3)
		4 ... 6 ... 3 = 12 \vspace{0.5em} \\
		% (8 - 5) × (2 + 2)
		8 ... 5 ... 2 ... 2 = 12 &
	\end{tabular}
	\vspace{1em}
\end{question}

\begin{question}[(3 points)]
	Farid dispose de 6 billets de 50 euros. Il achète un lecteur MP3 à 80 euros et cinq DVD à 9 euros l'unité.
	\begin{itemize}
		% 6 × 50 - (80 + 5 × 9)
		\item \textbf{En utilisant les nombres de l'énoncé}, écrit une expression représentant la somme restante après ses achats. \vspace{2em}
		      % 6 × 50 - (80 + 5 × 9)
		      % = 300 - (80 + 5 × 9)
		      % = 300 - (80 + 45)
		      % = 300 - 125
		      % = 175 €
		\item Effectue les calculs correspondants. \vspace{8em}
	\end{itemize}
\end{question}

\begin{question}[(4 points)]\

	La halle Tony-Garnier, à Lyon, peut recevoir 3000 personnes au total. 2000 places sont situées devant la scène et les autres un peu plus loin sur des gradins.

	Lors du premier concert, la salle est pleine. Les places devant la scène sont vendues à 24€ et les places en gradins à 10€.
	\begin{enumerate}
		% 3000 - 2000 = 1000
		\item Combien y-a-t'il de places en gradins ?
		      % (2000 × 24) + ((3000 - 2000) × 10)
		\item \textbf{En utilisant les nombres de l'énoncé}, écrit une expression permettant de calculer la recette du concert. \vspace{2em}
		      % 48 000 + 15 000 = 63 000
		\item Calcule cette recette. \vspace{12em}
	\end{enumerate}
\end{question}

\begin{question}[(2 points)]
	Ecrit chacune de ces phrases à l’aide d’un calcul, puis calcule le résultat :
	\begin{enumerate}
		% 4 × (9 + 5) = 56
		\item Le produit de quatre par la somme de neuf et de cinq. \vspace{3em}
		      % 6 × 8 + 20 = 68
		\item La somme du produit de six par huit et de vingt. \vspace{3em}
		      % 5 × 2 - 6 = 4
		\item La différence entre le produit de cinq par deux et six. \vspace{3em}
		      % (7 + 8) ÷ (21 - 6) = 1
		\item Le quotient de la somme de sept et de huit par la différence de vingt et un et de six.
	\end{enumerate}
\end{question}

\end{document}