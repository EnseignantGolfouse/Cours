\documentclass[a4paper,11pt]{article}

\usepackage{../préambule}

\title{Contrôle n°2 : Priorités opératoires}
\date{22 septembre 2021}
\author{Prénom : ..........}

\makeatletter
\renewcommand{\maketitle}{%
    \topskip1em
	\@author \hfill \@date \\

	\begin{center}
		\begin{huge}
			\@title \\[1em]
		\end{huge}
	\end{center}
}
\makeatother

\begin{document}

\maketitle

\begin{question}[(4 points)] Effectue les calculs suivants en respectant les priorités opératoires :

	\noindent
	\begin{tabular}{p{0.47\textwidth}|p{0.47\textwidth}}
		A = 39 − 14 + 42 − 7 & B = 15 − 5 × 3 + 6 \vspace{6em}  \\
		C = 24 ÷ 4 × 7 ÷ 2   & D = 45 − 3 × 2 × 6  \vspace{6em}
	\end{tabular}
\end{question}


\begin{question}[(4 points)] Effectue les calculs suivants en respectant les priorités opératoires :

	\noindent
	\begin{tabular}{p{0.47\textwidth}|p{0.47\textwidth}}
		E = 40 − (13 + 7)    & F = (4 + 1) × (6 + 3) \vspace{7em}      \\
		G = 3 + (14 − 5) × 3 & H = 37 − [3 × (5 + 2) − 4] \vspace{7em}
	\end{tabular}
\end{question}


\begin{question}[(3 points)] Complète en utilisant $+,-,×,÷$ pour que les égalités soient vraies :

	\vspace{1em}
	\noindent
	\begin{tabular}{p{0.47\textwidth}p{0.47\textwidth}}
		% 16 - 7 + 6
		16 ... 7 ... 6 = 15       &
		% 6 + 6 - 6 ÷ 6
		6 ... 6 ... 6 ... 6 = 11 \vspace{0.5em} \\
		% 4 + 7 × 12 - 8
		4 ... 7 ... 12 ... 8 = 80 &
	\end{tabular}
	\vspace{1em}

	Complète en utilisant $+,-,×,÷$ \textbf{et les parenthèses} pour que les égalités soient vraies :

	\vspace{1em}
	\noindent
	\begin{tabular}{p{0.47\textwidth}p{0.47\textwidth}}
		% (3 + 3) × 5
		3 ... 3 ... 5 = 30       &
		% 3 × (7 - 3)
		3 ... 7 ... 3 = 12 \vspace{0.5em} \\
		% (9 - 6) × (1 + 3)
		9 ... 6 ... 1 ... 3 = 12 &
	\end{tabular}
	\vspace{1em}
\end{question}

\begin{question}[(2 points)]
	Farid dispose de 6 billets de 50 euros. Il achète un lecteur MP3 à 70 euros et cinq DVD à 11 euros l'unité.
	\begin{itemize}
		% 6 × 50 - (70 + 5 × 11)
		\item \textbf{En utilisant les nombres de l'énoncé}, écrit une expression représentant la somme restante après ses achats. \vspace{2em}
		      % 6 × 50 - (70 + 5 × 11)
		      % = 300 - (70 + 5 × 11)
		      % = 300 - (70 + 55)
		      % = 300 - 125
		      % = 175 €
		\item Effectue les calculs correspondants. \vspace{6em}
	\end{itemize}
\end{question}

\begin{question}[(5 points)]\

	La halle Tony-Garnier, à Lyon, peut recevoir 5000 personnes au total. 2000 places sont situées devant la scène et les autres un peu plus loin sur des gradins.
	\begin{enumerate}
		\item
		      Lors du premier concert, la salle est pleine. Les places devant la scène sont vendues à 24€ et les places en gradins à 10€.
		      \begin{enumerate}
			      % 5000 - 2000 = 3000
			      \item Combien y-a-t'il de places en gradins ?
			            % (2000 × 24) + ((5000 - 2000) × 10)
			      \item \textbf{En utilisant les nombres de l'énoncé}, écrit en une expression permettant de calculer la recette du concert. \vspace{2em}
			            % 48 000 + 30 000 = 78 000
			      \item Calcule cette recette. \vspace{8em}
		      \end{enumerate}
		\item Lors du deuxième concert, les 2000 places devant la scène sont de nouveau vendues et la recette est de 68 000€.
		      \begin{enumerate}
			      % (68 000 - 2000 × 24) ÷ 10
			      \item \textbf{En utilisant les nombres de l'énoncé}, écrit en une expression le calcul du nombre de spectateurs dans les gradins. \vspace{2em}
			            % = (68 000 - 48 000) ÷ 10
			            % = 20 000 ÷ 10
			            % = 2000
			      \item Calcule le nombre de spectateurs dans les gradins. \vspace{8em}
		      \end{enumerate}
	\end{enumerate}
\end{question}

\begin{question}[(2 points)]
	Ecrit chacune de ces phrases à l’aide d’un calcul, puis calcule le résultat :
	\begin{enumerate}
		% 4 × (11 + 5) = 64
		\item Le produit de quatre par la somme de onze et de cinq. \vspace{3em}
		      % 6 × 7 + 20 = 62
		\item La somme du produit de six par sept et de vingt. \vspace{3em}
		      % 6 × 2 - 5 = 7
		\item La différence entre le produit de six par deux et cinq. \vspace{3em}
		      % (8 + 7) ÷ (18 - 3) = 1
		\item Le quotient de la somme de huit et de sept par la différence de dix-huit et de trois.
	\end{enumerate}
\end{question}

\end{document}