\documentclass[a4paper,11pt]{article}

\usepackage{préambule}

\title{Contrôle n°2 : Priorités opératoires (Bonus)}
\date{15 octobre 2021}
\author{Prénom : ..................}

\makeatletter
\renewcommand{\maketitle}{%
    \topskip2em
	\@author \hfill \@date \\

	\begin{center}
		\begin{huge}
			\@title \\[2em]
		\end{huge}
	\end{center}
}
\makeatother

\begin{document}

\maketitle

\begin{center}
	\large
	\textbf{Note tes recherches sur la feuille !}
\end{center}

\begin{question*}[Bonus]\

	Nous sommes deux nombres entiers. Notre somme est 20, notre produit est 96.

	Qui sommes-nous ?
	\vspace{10em}
	% 12 et 8
\end{question*}

\begin{question*}[Bonus]\

	George et Henri achètent des décorations pour Halloween.
	\begin{itemize}
		\item Au premier magasin, ils dépensent la moitié de tout leur argent plus 10 euros.
		\item Au deuxième magasin, ils dépensent le tiers de ce qui leur reste plus 8 euros.
		\item Au troisième magasin, ils dépensent le quart de ce qui leur reste.
		\item Au quatrième magasin, ils dépensent leur derniers 6 euros.
	\end{itemize}
	Combien avaient-ils d’argent à la base ?
	% Réponse :
	% - 6
	% - 1/4 x + 6 = x → x = 8
	% - 1/3 x + 8 + 8 = x → x = 24
	% - 1/2 x + 10 + 24 = x → 72€
\end{question*}


\end{document}