\documentclass[a4paper,11pt]{article}

\usepackage{../préambule}

\title{Évaluation}
\date{5 octobre 2021}


\begin{document}

\begin{minipage}[t][0.45\textheight][t]{\textwidth}
	Prénom: .......................... \hfill 5\textsuperscript{ème} 8
	\begin{center}
		\begin{huge}
			Évaluation
		\end{huge}

		5 octobre 2021
	\end{center}

	\begin{exercice}[(2 points)]
		Compléter le vocabulaire suivant :
		\begin{itemize}
			\item Dans l'expression $12 + 65 = 77$, $12$ et $65$ sont les .................... .
			\item Dans l'expression $17 × 5 = 85$, $85$ est le .................... de $17$ par $5$.
		\end{itemize}

		Entourer le mot qui correspond :

		\begin{center}
			Dans l'expression $6 = \frac{42}{7}$, $6$ est le quotient de $42$ ...... $7$.

			\begin{multicols}{4}
				\begin{itemize}
					\item de
					\item et de
					\item par
					\item sur
				\end{itemize}
			\end{multicols}
		\end{center}
	\end{exercice}

	\begin{exercice}[(6 points)]
		Faire les calculs en \textbf{détaillant} les étapes :

		\begin{tabular}{ccc}
			$A = 7 - 6 + 12$ \hspace{1cm} & $B = 13 × 2 + 7 × 8$ \hspace{1cm} & $C = 55 - 6 - 3 × 5$
		\end{tabular}
		\vspace{1cm}
	\end{exercice}

\end{minipage}

\hrule
\vspace{0.05\textheight}

\begin{minipage}[t][0.45\textheight][t]{\textwidth}

	\begin{exercice}[(2 points)]
		On veut calculer $3 × 6 - 4 ÷ 2$.
		\begin{itemize}
			\item Diego affirme avoir trouvé $7$. A-t-il raison ?

			      Quel calcul a-t-il effectué ? \vspace{3cm}
			\item Élise affirme avoir trouvé $16$. A-t-elle raison ?

			      Quel calcul a-t-elle effectué ?\vspace{3cm}
		\end{itemize}
	\end{exercice}

\end{minipage}

\end{document}