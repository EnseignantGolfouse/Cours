\documentclass[a4paper,12pt]{article}

\usepackage{../../préambule}
\usepackage{clipboard} % copie-colle du code latex

\title{Le défi de Zoé : bonus}
\date{}
\author{}

\begin{document}

\addtolength{\extrarowheight}{0.5em}

\maketitle

\begin{exercice}\
	\begin{center}
		\begin{tabularx}{10em}{
				| >{\centering\arraybackslash}X
				| >{\centering\arraybackslash}X
				| >{\centering\arraybackslash}X
				| >{\centering\arraybackslash}X
				| >{\centering\arraybackslash}X |}
			\hline
			30 & +  & 13 & - & 8  \\ \hline
			-  & 21 & -  & 7 & ×  \\ \hline
			5  & ×  & 4  & × & 3  \\ \hline
			+  & 6  & ×  & 2 & +  \\ \hline
			15 & ÷  & 3  & - & 11 \\ \hline
		\end{tabularx}
	\end{center}

	Comme dans le jeu de base, on commence en haut à gauche, et on ne peut avancer que vers la droite et le bas.

	\begin{itemize}
		\item Quel est le chemin qui donne le plus grand nombre possible ? % 337 = 30 + 21 × 6 × 3 – 11
		\item Comment obtenir 24 ? % 30 + 13 – 4 × 2 – 11
		\item Comment obtenir 9 ? % 30 – 5 × 6 ÷ 3 – 11
	\end{itemize}

\end{exercice}

\begin{exercice}
	Compléter la grille suivante pour que les 2 résultats possibles soient $60$ et $28$ :
	\begin{center}
		\begin{tabularx}{4em}{
				| >{\centering\arraybackslash}X
				| >{\centering\arraybackslash}X |}
			\hline
			30 & \\ \hline
			   & \\ \hline
		\end{tabularx}
	\end{center}

	Compléter la grille suivante pour que les résultats possibles soient 1, 10, 11, 16, 21, 25 :
	\begin{center}
		\begin{tabularx}{6em}{
				| >{\centering\arraybackslash}X
				| >{\centering\arraybackslash}X
				| >{\centering\arraybackslash}X |}
			\hline
			13 &   &   \\ \hline
			   & 6 & ÷ \\ \hline
			   &   & 2 \\ \hline
		\end{tabularx}
	\end{center}
	% 13 + 16
	% -  6 ÷
	% 1  × 2
\end{exercice}

\end{document}