\documentclass[a4paper,11pt]{article}

\usepackage{préambule}
\usepackage{clipboard} % copie-colle du code latex

% Pour entourer un nombre en rouge
\newcommand*\redcircled[1]{\tikz[baseline=(char.base)]{
	\node[shape=circle,draw,inner sep=2pt,draw=red] (char) {#1};
}}


\begin{document}

Prénom: {\color{red} CORRIGÉ} \hfill 5\textsuperscript{ème} 8
\begin{center}
	\begin{huge}
		Évaluation
	\end{huge}

	5 octobre 2021
\end{center}

\begin{exercice}[(2 points)]
	Compléter le vocabulaire suivant :
	\begin{itemize}
		\item Dans l'expression $12 + 65 = 77$, $12$ et $65$ sont les {\color{red} termes} .
		\item Dans l'expression $17 × 5 = 85$, $85$ est le {\color{red} produit} de $17$ par $5$.
	\end{itemize}

	Entourer le mot qui correspond :

	\begin{center}
		Dans l'expression $6 = \frac{42}{7}$, $6$ est le quotient de $42$ {\color{red} par} $7$.

		\begin{multicols}{4}
			\begin{itemize}
				\item de
				\item et de
				\item \redcircled{par}
				\item sur
			\end{itemize}
		\end{multicols}
	\end{center}
\end{exercice}

\begin{exercice}[(6 points)]
	Faire les calculs en \textbf{détaillant} les étapes :

	\begin{align*}
		A & = \uline[\color{red}]{7 − 6} + 12 & B & = \uline[\color{red}]{13 × 2} + 7 × 8 & C & = 55 - 6 - \uline[\color{red}]{3 × 5}
	\end{align*}
	\color{red}
	\begin{align*}
		A & = \uline{1 + 12} & B & = 26 + \uline{7 × 84} & C & = \uline{55 − 6} - 15 \\
		A & = 13               & B & = \uline{26 + 56}     & C & = \uline{49 − 15}     \\
		  &                    & B & =  82                   & C & = 34
	\end{align*}
\end{exercice}

\hrule
\vspace{0.05\textheight}

% \newcommand\exerciceCounterBackup{\numexpr\value{exercice}-1}
% \setcounter{exercice}{0}

% \begin{minipage}[t][0.45\textheight][t]{\textwidth}
% 	\Paste{recto}
% \end{minipage}

% \newpage

\begin{exercice}[(2 points)]
	On veut calculer $3 × 6 - 4 ÷ 2$.
	\begin{itemize}
		\item Diego affirme avoir trouvé $7$. A-t-il raison ?

		      Quel calcul a-t-il effectué ?

		      {\color{red}
		      Non, Diego n'a pas raison, car il a fait les calculs de \textbf{gauche à droite}, sans respecter les priorités :
		      \begin{align*}
			        & \uline{3 × 6} − 4 ÷ 2 \\
			      = & \uline{18 − 4} ÷ 2    \\
			      = & \uline{14 ÷ 2}        \\
			      = & 7
		      \end{align*}
		      }
		\item Élise affirme avoir trouvé $16$. A-t-elle raison ?

		      Quel calcul a-t-elle effectué ?

		      {\color{red}
		      Oui, Élise a raison, car elle a appliqué les priorités de calcul :
		      \begin{align*}
			        & \uline{3 × 6} − 4 ÷ 2 \\
			      = & 18 − \uline{4 ÷ 2}    \\
			      = & \uline{18 − 2}        \\
			      = & 16
		      \end{align*}
		      }
	\end{itemize}
\end{exercice}

% \hrule
% \vspace{0.05\textheight}

% \setcounter{exercice}{\exerciceCounterBackup}

% \begin{minipage}[t][0.45\textheight][t]{\textwidth}
% 	\Paste{verso}
% \end{minipage}

\end{document}