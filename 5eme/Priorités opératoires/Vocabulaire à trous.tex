\documentclass[a4paper]{article}

\usepackage{préambule}

% marges
\addtolength{\oddsidemargin}{-2.1cm}
\addtolength{\evensidemargin}{-2.1cm}
\addtolength{\textwidth}{4.2cm}
\addtolength{\topmargin}{-2cm}
\addtolength{\textheight}{4cm}

\begin{document}

\begin{cours}[Vocabulaire]
	\begin{itemize}
		\item Le résultat d'une addition est une ................ . Les nombres additionés sont les ................ .
		\item Le résultat d'une soustraction est une .................... . Les nombres qui interviennent dans la\\[0.3em] soustraction sont les ................ .
		\item Le résultat d'une multiplication est un ................ . Les nombres multipliés sont les ................ .
		\item Le résultat d'une division est un ................ .
		      \begin{itemize}
			      \item Si l'opération est écrite avec le signe “$÷$”, on dit qu'on divise un .................... par un\\[0.3em] ................ .
			      \item Si l'opération est écrite comme une fraction, on dit qu'on divise un ....................  par un\\[0.3em] .................... .
		      \end{itemize}
	\end{itemize} \vspace{1em}

	On fera attention au vocabulaire utilisé, notamment les prépositions (de, par, entre, ...). Regarde bien les exemples ci-dessous pour savoir quoi utiliser.
\end{cours}

\begin{exemple}
	\begin{itemize}
		\item Dans $6 + 3,2 = 9,2$ :
		      \begin{itemize}
			      \item $6$ et $3,2$ sont les .................. .
			      \item $9,2$ est la .................. \textbf{de} $6$ \textbf{et} $3,2$.
		      \end{itemize}
		\item Dans $8,7 - 6,5 = 2,2$ :
		      \begin{itemize}
			      \item $8,7$ et $6,5$ sont les .................. .
			      \item $2,2$ est la .................... \textbf{entre} $8,7$ \textbf{et} $6,5$.
		      \end{itemize}
		\item Dans $5 × 1,2 = 6$ :
		      \begin{itemize}
			      \item $5$ et $1,2$ sont les .................. .
			      \item $6$ est le .................. \textbf{de} $5$ \textbf{par} $1,2$.
		      \end{itemize}
		\item Dans $8 ÷ 5 = 1,6$ :
		      \begin{itemize}
			      \item $8$ est le ...................., $5$ est le .................. .
			      \item $1,6$ est le .................. \textbf{de} $8$ \textbf{par} $5$.
		      \end{itemize}
		\item Dans $\frac{6}{4} = 1,5$ :
		      \begin{itemize}
			      \item $6$ est le ...................., $5$ est le ...................... .
			      \item $1,5$ est le .................. \textbf{de} $6$ \textbf{par} $4$.
		      \end{itemize}
	\end{itemize}
\end{exemple}

\end{document}