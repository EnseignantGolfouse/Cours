\documentclass[a4paper,10pt]{article}

\usepackage{préambule}
\usetikzlibrary{angles,arrows,arrows.meta,calc,intersections,quotes}

% Usage : create a point
% (parallel cs:from=P,to=A--B,on=C--D)
% The projection of P onto A--B parallel to C--D.
\makeatletter
\tikzset{/tikz/parallel cs/.cd,
    to line initial coordinate/.store in=\tikz@parallelcs@toA,
    to line final coordinate/.store in=\tikz@parallelcs@toB,
    on line initial coordinate/.store in=\tikz@parallelcs@onA,
    on line final coordinate/.store in=\tikz@parallelcs@onB,
    from coordinate/.store in=\tikz@parallelcs@from,
    to/.style args={#1--#2}{
        to line initial coordinate=#1,
        to line final coordinate=#2,
    },
    on/.style args={#1--#2}{
        on line initial coordinate=#1,
        on line final coordinate=#2,
    },
    from/.style={
        from coordinate=#1
    }
}
\tikzdeclarecoordinatesystem{parallel}{
    \tikzset{/tikz/parallel cs/.cd,#1}
    \tikz@scan@one@point\pgfutil@firstofone%
    (intersection of \tikz@parallelcs@from --$(\tikz@parallelcs@toA)!(\tikz@parallelcs@from)!90:(\tikz@parallelcs@toB)$ and \tikz@parallelcs@onA--\tikz@parallelcs@onB)%
    \relax
}
\makeatother


\makeatletter
\renewcommand{\maketitle}{%
{\scriptsize colle dans ton cahier d'exercices}
	\begin{center}
		\LARGE
		\myuline{\@title}
		\vspace{0.5em}
	\end{center}
}
\makeatother

\title{Activité : périmètre de la terre}
\date{}
\author{}

\begin{document}

\maketitle

\begin{tikzpicture}
	\coordinate (Centre de la Terre) at (0,0);
	\coordinate (Soleil1) at (10,3);
	\coordinate (Soleil2) at (10,6);
	\node[right] at (10,4) {Soleil};

	\draw[name path=Terre] (Centre de la Terre) circle(4cm);
	\draw[name path=Rayon1] (Soleil1) -- (Centre de la Terre);
	\path[name intersections={of= Terre and Rayon1,by=Syène}];

	\coordinate (Centre de la Terre décalé) at ($(Centre de la Terre) + (Soleil2) - (Soleil1)$);
	\path[name path=Rayon2] (Soleil2) -- (Centre de la Terre décalé);

	\path[name intersections={of= Terre and Rayon2,by=H}];
	\draw (Soleil2) -- (H);
	
	\coordinate[rotate around={24:(Centre de la Terre)}] (RotateSoleil1) at (Soleil1);
	\path[name path=HauteurAlexandrie] (Centre de la Terre) -- (RotateSoleil1);
	\path[name intersections={of= HauteurAlexandrie and Rayon2,by=Obélisque}];
	\path[name intersections={of= HauteurAlexandrie and Terre,by=Alexandrie}];

	\draw (Centre de la Terre) -- (Alexandrie);
	\draw[ultra thick] (Alexandrie) -- (Obélisque);

	\draw pic[draw,fill=red!60,angle radius=1cm,"7.12" shift={(-11mm,-5mm)}] {angle=H--Obélisque--Alexandrie};

	\foreach \p/\dir in {Centre de la Terre/below,Syène/below right,H/above,Alexandrie/right,Obélisque/above} {
			\node at (\p) {∙};
			\node[\dir] at (\p) {\p};
		}
\end{tikzpicture}

\end{document}